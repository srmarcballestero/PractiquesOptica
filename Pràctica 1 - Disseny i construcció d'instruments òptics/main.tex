\documentclass[11pt,a4paper,catalan]{article}
\author{Marc Ballestero Ribó}

%___PAQUETS NECESSARIS___%
\usepackage[utf8]{inputenc}
\usepackage{amsthm,amssymb,amsmath,mathrsfs,babel}
\usepackage{wasysym}
\usepackage[margin=25mm]{geometry}
\usepackage[bottom]{footmisc}
\usepackage{enumerate}
\usepackage{array, boldline, makecell, booktabs, dcolumn}
\usepackage[svgnames, table]{xcolor}
\usepackage[all]{xy}
\usepackage{parskip}
\usepackage{float}
\usepackage{fancyhdr}
\usepackage{graphicx}
\usepackage{multirow,multicol}
\usepackage{pgfplots, tikz}
\usepackage{pgfplotstable}
\usepackage{graphics}
\usepackage{arydshln}
\usepackage{caption}
\usepackage{siunitx}
\usepackage{chngpage}
\usepackage{enumitem}
\usepackage{hyperref}
\hypersetup{
    colorlinks=true,
    linkcolor=black,
    filecolor=black,
    urlcolor=blue,
}

\sisetup{separate-uncertainty=true, exponent-product=\cdot}
\DeclareSIUnit\dioptre{D}
\decimalpoint



%\sisetup{locale = FR}

%___NOUS TIPUS DE COLUMNES PER LES TAULES___%
\newcolumntype{L}[1]{>{\raggedright\let\newline\\\arraybackslash\hspace{0pt}}m{#1}}
\newcolumntype{C}[1]{>{\centering\let\newline\\\arraybackslash\hspace{0pt}}m{#1}}
\newcolumntype{R}[1]{>{\raggedleft\let\newline\\\arraybackslash\hspace{0pt}}m{#1}}
\newcolumntype{d}[1]{D{.}{.}{#1}}

%___COMANDAMENTS PERSONALITZATS___%
\providecommand{\keywords}[1]{\textsc{{Paraules clau---}} #1}
\newcommand{\diff}{\mathrm{d}}
\newcommand{\nind}{\noindent}
\newcommand{\res}[3]{$\left ( #1 \pm #2 \right ) \, \si{#3}$}
\newcommand{\figref}[1]{figura \ref{#1}}
\newcommand{\tabref}[1]{tabla \ref{#1}}
\renewcommand\thesection{\arabic{section}}
\newcommand{\data}{\today}


%___CAPÇALERA I PEU DE PÀGINA___%
\pagestyle{fancy}
\headheight=16pt
\setlength{\headsep}{1cm}
\fancypagestyle{headings}{
\cfoot{\thepage}}
\lhead{\scshape Òptica \textup{2020-21}}
\rhead{\data \\ Marc Ballestero Ribó}

%___CONFIGURACIÓ DEL PGFPLOTS___%
\pgfplotsset{compat = 1.15}
\pgfplotsset{
    legend image with text/.style={
        legend image code/.code={%
            \node[anchor=center] at (0.3cm,0cm) {#1};
        }
    },
    log x ticks with fixed point/.style={
      xticklabel={
        \pgfkeys{/pgf/fpu=true}
        \pgfmathparse{exp(\tick)}%
        \pgfmathprintnumber[fixed relative, precision=0]{\pgfmathresult}
        \pgfkeys{/pgf/fpu=false}
      }
    },
    log y ticks with fixed point/.style={
      yticklabel={
        \pgfkeys{/pgf/fpu=true}
        \pgfmathparse{exp(\tick)}%
        \pgfmathprintnumber[fixed zerofill, precision=0]{\pgfmathresult}
        \pgfkeys{/pgf/fpu=false}
      }
  },
}
\usetikzlibrary{spy}
%___CONFIGURACIÓ DELS PEUS DE FIGURA___%
\captionsetup[figure]{justification = centering,labelfont={small,sc},textfont=small}
\captionsetup[table]{name = Tabla, justification = centering,labelfont={small,sc},textfont=small}




%___TÍTOL___%

\title{\vspace{-1.5cm} Òptica \\ \vspace{1mm} Práctica 1. Disseny i construcció d'instruments òptics.}
\author{Marc Ballestero Ribó - Grup D2}
\date{\data}


\begin{document}
\maketitle

\begin{abstract}
\nind El objetivo de esta práctica es estudiar las oscilaciones amortiguadas y forzadas de un péndulo de Pohl. Primeramente, se ha estudiado el comportamiento del péndulo cuando se le aplica una fuerza de fricción dependiente de su velocidad de rotación, dada por un freno magnético. En este caso se ha obtenido que el péndulo se comporta como un oscilador armónico débilmente amortiguado, con constante de amortiguación $\beta = \SI{0.177\pm0.007}{\per\second}$. Una vez conocido el comportamiento del péndulo ante la fuerza dada por el freno magnético, se ha aplicado un forzamiento periódico al sistema, dado por un motor de potencia regulable. Estudiando el comportamiento estacionario del péndulo para distintas frecuencias de forzamiento, se ha conseguido aproximar el valor de la frecuencia de resonancia en amplitud del sistema, dado por $\omega_\textup{R} =\SI{2.855  \pm 0.003}{\per\second}$.

\end{abstract}


\section{Determinació de la potència de les lents} \label{sec:potlens}
Per a determinar la potència de les diferents lents usades en la pràctica, s'ha usat un frontofocòmetre, aparell que ens permet determinar aquest paràmetre, així com el centre òptic i la curvatura de les lents, tot enfocant un reticle prefixat. Atès que tots els experiments s'han realitzat sota el supòsit de l'aproximació paraxial, l'única dada que s'ha mesurat de les lents és la seva potència $P$. La incertesa associada a aquesta mesura depèn del propi valor de $P$ segons
\begin{equation}
    \delta P =
        \begin{cases}
            \SI{0.13}{\dioptre}, & \text{si } P \sim \SI{1}{\dioptre} \\
            \SI{0.3}{\dioptre}, & \text{si } P \sim \SI{10}{\dioptre}
        \end{cases}
\end{equation}

\section{Ullera astronòmica} \label{sec:ullastro}
Una ullera astronòmica és un telescopi que consta de dues lents convergents, anomenades \emph{objectiu} i \emph{ocular} posicionades de tal manera que coincideixin el pla focal imatge de l'objectiu i el pla focal objecte de l'ocular. L'objectiu és alhora el diafragma d'obertura (DA) i la pupil·la d'entrada (PE), i la seva imatge a través de l'ocular ens dóna la pupil·la de sortida (PS).

Per a la construcció de la ullera astronòmica hem usat, doncs, dues lents convergents, una de potència $P_\textup{obj} = \SI{2.00 \pm 0.13}{\dioptre}$ per a l'objectiu i una de $P_\textup{ocu} = \SI{9.5 \pm 0.3}{\dioptre}$ per a l'ocular, i de diàmetres respectius $\diameter_\textup{obj} = \SI{6.5 \pm 0.1}{\centi\metre}$ i $\diameter_\textup{ocu} = \SI{5.5 \pm 0.1}{\centi\metre}$.

Les focals de les lents vindran donades per
\begin{equation}
    f'_\textup{i} = \frac{1}{P_\textup{i}}, \hspace{2mm} \delta f'_\textup{i} = \frac{\delta P_\textup{i}}{P_\textup{i}^2}
\end{equation}
d'on obtenim que $f'_\textup{obj} = \SI{52 \pm 3}{\centi\metre}$ i $f'_\textup{ocu} = \SI{10.5 \pm 0.3}{\centi\metre}$.

L'emergència de la pupil·la de sortida $e$, amb la seva corresponent incertesa $\delta e$, la podem calcular segons
\begin{gather}
    - \frac{1}{-\left (f'_\textup{obj} + f'_\textup{ocu}\right)} + \frac{1}{e} = \frac{1}{f'_\textup{ocu}} \Rightarrow e = \frac{f'_\textup{ocu}}{f'_\textup{obj}}\left(f'_\textup{obj}+f'_\textup{ocu}\right) \\
    \delta e = \sqrt{\left(\frac{f'_\textup{ocu}}{f'_\textup{obj}}\right)^4\left(\delta f'_\textup{obj}\right)^2 + \left[\left(2\,\frac{f'_\textup{ocu}}{f'_\textup{obj}} + 1\right) \delta f'_\textup{ocu}\right]^2}
\end{gather}
i obtenim que $e$ = \SI{12.6 \pm 0.4}{\centi\metre}.

Per calcular l'augment fem servir que
\begin{equation}
    \Gamma' = -\frac{f'_\textup{obj}}{f'_\textup{ocu}}, \hspace{2mm} \delta \Gamma' = \sqrt{\left(\frac{\delta f'_\textup{obj}}{f'_\textup{ocu}}\right)^2 + \left[\frac{f'_\textup{obj}}{\left(f'_\textup{ocu}\right)^2} \, \delta f'_\textup{ocu}\right]^2}
\end{equation}

El camp d'observació $\omega$ ve donat per
\begin{equation}
    \omega = \arctan{\frac{\diameter_\textup{DC}}{2\,e}}
\end{equation}
on $\diameter_\textup{DC}$ és el diametre del diafragma de camp.



\appendix
\newpage
\section{Fórmules estadístiques}\label{sec:apendix}
\setcounter{equation}{0}
\renewcommand{\theequation}{\thesection.\arabic{equation}}
%\renewcommand{\thesubsection}{\thesection.\Roman{subsection}}
\subsection{Paràmetres centrals i de dispersió}
Donada una mostra de $N$ elements $\left\{x_1,\dots,x_N\right\}$, definim els següents paràmetres estadístics.
\begin{itemize}[label=--]
    \item Mitjana aritmètica \begin{equation}\label{est:mitjana}
            \langle x\rangle =\frac{1}{N} \sum_{i=1}^{N} x_i
        \end{equation}
    \item Desviació estàndard\footnote{S'ha fet servir la correcció de Bessel $\sqrt{N/(N-1)}$ de la desviació estàndard poblacional.} \begin{equation}\label{est:stdev}
            {\displaystyle \sigma_x ={\sqrt {{\frac {1}{N-1}}\sum _{i=1}^{N}(x_{i}-\langle x \rangle )^{2}}}}
        \end{equation}
    \item Error estàndard \begin{equation}\label{est:err}
   {\displaystyle \delta{x} ={\frac {\sigma_x}{\sqrt {N}}}}
\end{equation}

\end{itemize}
\subsection{Estimacions lineals}
Per al càlcul de les estimacions lineals s'usa la funció \texttt{ESTIMACION.LINEAL} del full de càlcul \texttt{Microsoft Excel}, que proporciona el pendent i l'ordenada a l'origen de la recta de regressió amb les seves corresponents incerteses, així com el coeficient de correlació $R^2$ i l'error estàndard de la regressió.
% El funcionament complet de la funció es pot consultar a \url{https://support.office.com/es-es/article/estimacion-lineal-funci\%C3\%B3n-estimacion-lineal-84d7d0d9-6e50-4101-977a-fa7abf772b6d}.

\subsection{Test \texorpdfstring{$\chi^2$}{x2}}
Donat un ajust lineal $y=A\,x+B$, amb incertesa en la variable dependent $\delta y$ i error estàndard de la regressió $\delta y_\textup{reg}$, es defineix el \emph{coeficient $\chi^2$} com
\begin{equation}
    \chi^2 = \nu\, \left(\frac{\delta y_\textup{reg}}{\delta y}\right)
\end{equation}
on $\nu$ es el nombre de graus de llibertat de l'ajust. Amb això, podem definir el \emph{coeficient reduït $\chi^2_\nu$} com
\begin{equation}
    \chi^2_\nu = \frac{\delta y_\textup{reg}}{\delta y}
\end{equation}
El valor d'aquest paràmetre ens indica la bondat de l'ajust realitzat. Tenim que
\begin{enumerate}[label=(\alph*)]
    \item si \underline{$\delta y_\textup{reg} \ll \delta y$} o \underline{$\chi^2_\nu \ll 1$}, l'ajust és acceptable i probablement s'hagi sobreestimat $\delta y$;
    \item si \underline{$\delta y_\textup{reg} \lesssim \delta y$} o \underline{$\chi^2_\nu \lesssim 1$}, l'ajust és acceptable; i
    \item si \underline{$\delta y_\textup{reg} \gg \delta y$} o \underline{$\chi^2_\nu \gg 1$} l'ajust no és acceptable.
\end{enumerate}

\end{document}
