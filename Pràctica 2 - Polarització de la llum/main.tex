\documentclass[11pt,a4paper,catalan]{article}
\author{Marc Ballestero Ribó}

%___PAQUETS NECESSARIS___%
\usepackage[utf8]{inputenc}
\usepackage{amsthm,amssymb,amsmath,mathrsfs,babel}
\usepackage{wasysym}
\usepackage[margin=25mm]{geometry}
\usepackage[bottom]{footmisc}
\usepackage{enumerate}
\usepackage{array, boldline, makecell, booktabs, dcolumn}
\usepackage[svgnames, table]{xcolor}
\usepackage[all]{xy}
\usepackage{parskip}
\usepackage{float}
\usepackage{fancyhdr}
\usepackage{graphicx}
\usepackage{multirow,multicol}
\usepackage{pgfplots, tikz}
\usepackage{pgfplotstable}
\usepackage{graphics}
\usepackage{arydshln}
\usepackage{caption}
\usepackage{siunitx}
\usepackage{chngpage}
\usepackage{enumitem}
\usepackage{hyperref}
\hypersetup{
    colorlinks=true,
    linkcolor=black,
    filecolor=black,
    urlcolor=blue,
}

\sisetup{separate-uncertainty=true, exponent-product=\cdot}
\DeclareSIUnit\dioptre{D}
\decimalpoint



%\sisetup{locale = FR}

%___NOUS TIPUS DE COLUMNES PER LES TAULES___%
\newcolumntype{L}[1]{>{\raggedright\let\newline\\\arraybackslash\hspace{0pt}}m{#1}}
\newcolumntype{C}[1]{>{\centering\let\newline\\\arraybackslash\hspace{0pt}}m{#1}}
\newcolumntype{R}[1]{>{\raggedleft\let\newline\\\arraybackslash\hspace{0pt}}m{#1}}
\newcolumntype{d}[1]{D{.}{.}{#1}}

%___COMANDAMENTS PERSONALITZATS___%
\providecommand{\keywords}[1]{\textsc{{Paraules clau---}} #1}
\newcommand{\diff}{\mathrm{d}}
\newcommand{\nind}{\noindent}
\newcommand{\res}[3]{$\left ( #1 \pm #2 \right ) \, \si{#3}$}
\newcommand{\figref}[1]{figura \ref{#1}}
\newcommand{\tabref}[1]{tabla \ref{#1}}
\renewcommand\thesection{\arabic{section}}
\newcommand{\data}{\today}


%___CAPÇALERA I PEU DE PÀGINA___%
\pagestyle{fancy}
\headheight=16pt
\setlength{\headsep}{1cm}
\fancypagestyle{headings}{
\cfoot{\thepage}}
\lhead{\scshape Òptica \textup{2020-21}}
\rhead{\data \\ Marc Ballestero Ribó}

%___CONFIGURACIÓ DEL PGFPLOTS___%
\pgfplotsset{compat = 1.15}
\pgfplotsset{
    legend image with text/.style={
        legend image code/.code={%
            \node[anchor=center] at (0.3cm,0cm) {#1};
        }
    },
    log x ticks with fixed point/.style={
      xticklabel={
        \pgfkeys{/pgf/fpu=true}
        \pgfmathparse{exp(\tick)}%
        \pgfmathprintnumber[fixed relative, precision=0]{\pgfmathresult}
        \pgfkeys{/pgf/fpu=false}
      }
    },
    log y ticks with fixed point/.style={
      yticklabel={
        \pgfkeys{/pgf/fpu=true}
        \pgfmathparse{exp(\tick)}%
        \pgfmathprintnumber[fixed zerofill, precision=0]{\pgfmathresult}
        \pgfkeys{/pgf/fpu=false}
      }
  },
}
\usetikzlibrary{spy}

%___CONFIGURACIÓ DELS PEUS DE FIGURA___%
\captionsetup[figure]{justification = centering,labelfont={small,sc},textfont=small}
\captionsetup[table]{name = Tabla, justification = centering,labelfont={small,sc},textfont=small}

%___DIRECTORI D'IMATGES___%
\graphicspath{ {./Imatges/} }




%___TÍTOL___%

\title{\vspace{-1.5cm} Òptica \\ \vspace{1mm} Práctica 1. Disseny i construcció d'instruments òptics.}
\author{Marc Ballestero Ribó - Grup D2}
\date{\data}


\begin{document}
\maketitle

\ifx
\begin{abstract}
\nind El objetivo de esta práctica es estudiar las oscilaciones amortiguadas y forzadas de un péndulo de Pohl. Primeramente, se ha estudiado el comportamiento del péndulo cuando se le aplica una fuerza de fricción dependiente de su velocidad de rotación, dada por un freno magnético. En este caso se ha obtenido que el péndulo se comporta como un oscilador armónico débilmente amortiguado, con constante de amortiguación $\beta = \SI{0.177\pm0.007}{\per\second}$. Una vez conocido el comportamiento del péndulo ante la fuerza dada por el freno magnético, se ha aplicado un forzamiento periódico al sistema, dado por un motor de potencia regulable. Estudiando el comportamiento estacionario del péndulo para distintas frecuencias de forzamiento, se ha conseguido aproximar el valor de la frecuencia de resonancia en amplitud del sistema, dado por $\omega_\textup{R} =\SI{2.855  \pm 0.003}{\per\second}$.

\end{abstract}
\fi

\section{Determinació de la potència de les lents} \label{sec:potlens}
Per a determinar la potència de les diferents lents usades en la pràctica, s'ha usat un frontofocòmetre, aparell que ens permet determinar aquest paràmetre, així com el centre òptic i la curvatura de les lents, tot enfocant un reticle prefixat. Atès que tots els experiments s'han realitzat sota el supòsit de l'aproximació paraxial, l'única dada que s'ha mesurat de les lents és la seva potència $P$. La incertesa associada a aquesta mesura depèn del propi valor de $P$ segons
\begin{equation} \label{eq:potencia}
    \delta P =
        \begin{cases}
            \SI{0.13}{\dioptre}, & \text{si } P \sim \SI{1}{\dioptre} \\
            \SI{0.3}{\dioptre}, & \text{si } P \sim \SI{10}{\dioptre}
        \end{cases}
\end{equation}

\section{Ullera astronòmica} \label{sec:ullastro}
Una ullera astronòmica és un telescopi que consta de dues lents convergents, anomenades \emph{objectiu} i \emph{ocular} posicionades de tal manera que coincideixin el pla focal imatge de l'objectiu i el pla focal objecte de l'ocular. L'objectiu és alhora el diafragma d'obertura (DA) i la pupil·la d'entrada (PE), i la seva imatge a través de l'ocular ens dóna la pupil·la de sortida (PS).

\begin{figure}[H]
\centering
  \includegraphics[scale=0.8]{UlleraAstronomica.PNG}
  \captionof{figure}{Traçat de raigs per a la ullera astronòmica.}
\label{fig:ulleraastro}
\end{figure}


Per a la construcció de la ullera astronòmica hem usat, doncs, dues lents convergents, una de potència $P_\textup{obj} = \SI{2.00 \pm 0.13}{\dioptre}$ per a l'objectiu i una de $P_\textup{ocu} = \SI{9.5 \pm 0.3}{\dioptre}$ per a l'ocular, i de diàmetres respectius $\diameter_\textup{obj} = \SI{6.5 \pm 0.1}{\centi\metre}$ i $\diameter_\textup{ocu} = \SI{5.5 \pm 0.1}{\centi\metre}$.

Les focals de les lents vindran donades per
\begin{equation}\label{eq:focals}
    f'_\textup{i} = \frac{1}{P_\textup{i}}, \hspace{3mm} \delta f'_\textup{i} = \frac{\delta P_\textup{i}}{P_\textup{i}^2}
\end{equation}
d'on obtenim que $f'_\textup{obj} = \SI{50 \pm 3}{\centi\metre}$ i $f'_\textup{ocu} = \SI{10.5 \pm 0.3}{\centi\metre}$.

Per al muntatge del sistema, hem ajustat la posició de les lents observant en quina disposició les imatges a través del telescopi està enfocada. La distància entre les lents mesurada amb una cinta mètrica és $d = \SI{60.0\pm0.1}{\centi\metre}$, valor que observem que és compatible amb la suma de les focals de les lents $f'_\textup{obj} + f'_\textup{ocu} = \SI{63 \pm 3}{\centi\metre}$.

L'emergència de la pupil·la de sortida $e$, amb la seva corresponent incertesa $\delta e$, la podem calcular fent servir l'equació de les lents, tenint en compte que la pupil·la de sortida és la imatge de l'objectiu a través de l'ocular. Tenim que
\begin{gather}\label{eq:pupilasortida}
    - \frac{1}{-\left (f'_\textup{obj} + f'_\textup{ocu}\right)} + \frac{1}{e} = \frac{1}{f'_\textup{ocu}} \Rightarrow e = \frac{f'_\textup{ocu}}{f'_\textup{obj}}\left(f'_\textup{obj}+f'_\textup{ocu}\right)% \\
    %\delta e = \sqrt{\left(\frac{f'_\textup{ocu}}{f'_\textup{obj}}\right)^4\left(\delta f'_\textup{obj}\right)^2 + \left[\left(2\,\frac{f'_\textup{ocu}}{f'_\textup{obj}} + 1\right) \delta f'_\textup{ocu}\right]^2}
\end{gather}
i per tant $e$ = \SI{12.7 \pm 0.4}{\centi\metre}.

Per calcular l'augment fem servir que
\begin{equation}\label{eq:augments}
    \Gamma' = -\frac{f'_\textup{obj}}{f'_\textup{ocu}}%, \hspace{3mm} \delta \Gamma' = \sqrt{\left(\frac{\delta f'_\textup{obj}}{f'_\textup{ocu}}\right)^2 + \left[\frac{f'_\textup{obj}}{\left(f'_\textup{ocu}\right)^2} \, \delta f'_\textup{ocu}\right]^2}
\end{equation}
i obtenim que val $\Gamma' = -4.8 \pm 0.3$.

Ara, col·loquem un diafragma iris en diverses posicions i analitzem el comportament del sistema òptic.
Si posicionem el diafragma en el pla focal comú entre les dues lents, aquest actua com a diafragma de camp (DC). En aquest cas, observem que a mesura que reduïm l'obertura del diafragma es perd camp d'observació, però no il·luminació. El semicamp d'observació $\omega$ ve donat per
\begin{equation}\label{eq:semicamp}
    \omega = \arctan{\frac{\diameter_\textup{DC}}{2\,f'_\textup{obj}}}%, \hspace{3mm} \delta\omega = \frac{2}{\diameter_\textup{DC} ^2 + \left(2 f'_\textup{obj}\right)^2} \, \sqrt{\left(f'_\textup{obj} \delta \diameter_\textup{DC}\right)^2 + \left(\diameter_\textup{DC} \delta f'_\textup{obj} \right)^2}
\end{equation}
on $\diameter_\textup{DC}$ és el diametre del diafragma de camp, que hem près com $\diameter_\textup{DC} = \SI{2.5 \pm 0.1}{\centi\metre}$. Per tant, tenim que el camp d'observació val $2\omega = \SI{0.050\pm0.003}{\radian}$.

Per contra, observem que si col·loquem el diafragma just davant de l'objectiu no es perd camp d'observació però sí que es redueix la il·luminació de la imatge; cosa que s'explica tenint en compte que el diafragma limita la quantitat de llum que entra al sistema. El mateix passa si col·loquem el diafragma en la pupil·la de sortida, vist que aquesta és la imatge de la pupil·la d'entrada, que en aquest cas és l'objectiu.

Finalment, canviem l'ocular per un de potència $\tilde{P}_\textup{ocu} = \SI{20.0 \pm 0.3}{\dioptre}$ i diàmetre $\tilde{\diameter}_\textup{ocu} = \SI{5.0 \pm 0.1}{\centi\metre}$. Fent ús de les fórmules de la \eqref{eq:focals} a la \eqref{eq:augments}, obtenim que en aquest cas, l'emergència de la pupil·la de sortida és $\tilde{e} = \SI{5.50 \pm 0.10}{\centi\metre}$, l'augment $\tilde{\Gamma '} = -10.0 \pm 0.6$. La col·locació d'un diafragma en les mateixes posicions que en el cas anterior exhibeix un comportament anàleg i, en particular, inserint-ne un de $\diameter_\textup{DC} = \SI{2.5 \pm 0.1}{\centi\metre}$ en el pla focal comú es té el mateix camp d'observació que en el cas anterior, vist que l'equació \eqref{eq:semicamp} no depèn de la focal de l'ocular.

En tots dos casos la imatge a través del sistema presenta aberració cromàtica, causada pel fet que l'índex de refracció de la llum en un medi material depèn de la seva longitud d'ona, per tant les lents del sistema dispersen la llum que s'hi transmet.

\section{Ullera de Galileu} \label{sec:ulleragalileu}
Una ullera de Galileu és un telescopi en el qual l'objectiu és una lent convergent i l'ocular una lent divergent, amb la qual cosa s'aconsegueix un augment angular positiu. En aquest cas la imatge de la pupil·la d'entrada es forma a l'interior del telescopi, de manera que es limita el camp i esdevé lluerna de sortida (LS), actuant l'objectiu com a diafragma de camp (DC). La pupil·la de sortida (PS) és la pupil·la de l'ull de l'observador.

\begin{figure}[H]
\centering
  \includegraphics[scale=0.8]{UlleraGalileu.PNG}
  \captionof{figure}{Traçat de raigs per a la ullera de Galileu.}
\label{fig:ulleragalileu}
\end{figure}


Per a construir la ullera de Galileu hem usat un objectiu de potència $P_\textup{obj} = \SI{2.25 \pm 0.13}{\dioptre}$ i diàmetre $\diameter_\textup{obj} = \SI{6.5 \pm 0.1}{\centi\metre}$, i un ocular de potència $P_\textup{ocu} = \SI{-8.3 \pm 0.3}{\dioptre}$ i diàmetre $\diameter_\textup{ocu} = \SI{5.5 \pm 0.1}{\centi\metre}$. Usant \eqref{eq:focals} obtenim les focals, $f'_\textup{obj} = \SI{44 \pm 3}{\centi\metre}$ i $f'_\textup{ocu} = \SI{-12.0 \pm 0.4}{\centi\metre}$.

Per a muntar el sistema hem procedit de manera similar a la secció \ref{sec:ullastro}, comprovant en quina posició les imatges observades a través del telescopi s'enfoquen, obtenint que la separació adequada entre les lents és $d = \SI{33 \pm 0.1}{\centi\metre}$, valor compatible amb la suma de les focals $f'_\textup{obj} + \left | f'_\textup{ocu} \right | = \SI{32 \pm 3}{\centi\metre}$.

Fent servir les fórmules \eqref{eq:pupilasortida} i \eqref{eq:augments} obtenim que, en aquest cas, l'emergència de la pupil·la de sortida és $e = \SI{-15.3 \pm 0.7}{\centi\metre}$ i els augments valen $\Gamma' = 3.7 \pm 0.3$. Notem que per a aquest telescopi es compleix que, efectivament, $e < 0$, per tant la imatge de la pupil·la d'entrada es forma dins del tub, i $\Gamma' > 0$, per tant la imatge no s'inverteix.

Per calcular el camp d'observació, notem que en aquest cas tenim que $\diameter_\textup{DC} = \diameter_\textup{obj}$, vist que l'objectiu actua com a diafragma de camp limitant la quantitat de raigs que entren al sistema. Suposant que l'ull de l'observador es troba el més a prop possible de l'ocular i aplicant que
\begin{equation} \label{eq:semicampgalileo}
  \omega = \frac{1}{\Gamma'}\arctan{\frac{\diameter_\textup{PS}}{2\,\left |e\right |}}
\end{equation}
tenint en compte que el diàmetre de la pupil·la de sortida ve donat per $\diameter_\textup{PS} = \diameter_\textup{DC}/\left |\Gamma'\right|$, obtenim que el camp d'observació és $2\omega = \SI{0.032 \pm 0.006}{\radian}$. Evidentment, col·locant un diafragma iris de diàmetre inferior al davant de l'objectiu, limitem el camp i la il·luminació del sistema.

Finalment, cal remarcar que a les imatges a través d'aquest telescopi presenten una petita aberració esfèrica en els punts imatge allunyats del centre de la lent.

\section{Microscopi compost} \label{sec:microscopi}
\subsection{Il·luminació Köhler}\label{sec:kohler}
El sistema d'il·luminació Köhler ens permet, a partir d'una font de llum, il·luminar l'objecte a observar a través d'un microscopi amb un feix de llum col·limada. Això s'aconsegueix mitjançant dues lents convergents, anomenades \emph{condensador} i \emph{lent de camp} o \emph{col·limador}, disposades de manera que el colimador se situa al pla on es forma la imatge del condensador, la qual cosa fa que la llum provinent de la font surti del sistema paral·lela, uniforme i col·limada.

\begin{figure}[H]
\centering
  \includegraphics[scale=0.8]{Kohler.PNG}
  \captionof{figure}{Traçat de raigs per a la il·luminació kohler.}
\label{fig:kohler}
\end{figure}


Per a la construcció del sistema Köhler s'ha usat com a condensador una lent de potència $P_\textup{cond} = \SI{19.5 \pm 0.3}{\dioptre}$ i diàmetre $\diameter_\textup{cond} = \SI{4.0 \pm 0.1}{\centi\metre}$ i com a col·limador una lent de potència $P_\textup{col} = \SI{9.5 \pm 0.3}{\dioptre}$ i diàmetre $\diameter_\textup{col} = \SI{5.5 \pm 0.1}{\centi\metre}$. Mitjançant \eqref{eq:focals} obtenim les focals de les lents, $f'_\textup{cond} = \SI{5.13 \pm 0.08}{\centi\metre}$ i $f'_\textup{col} = \SI{10.5 \pm 0.3}{\centi\metre}$. S'ha escollit una lent de focal curta per al condensador per a poder captar el màxim con de llum provinent de la font. A més a més, s'ha equipat la font de llum amb un difusor per tal de reduir la il·luminació del sistema.

Per al muntatge del sistema, s'ha col·locat el condensador tan a prop de la font com és possible. Mitjançant una pantalla, s'ha observat en quina posició es forma la imatge de la font a través del condensador, i s'ha col·locat el col·limador de manera que el seu pla focal objecte coincideixi amb la posició determinada amb la pantalla.

\subsection{Microscopi}\label{sec:submicroscopi}
Un microscopi compost està format per dues lents convergents, anomenades \emph{objectiu} i \emph{ocular}, de manera que l'objectiu forma una imatge engrandida de l'objecte, i l'ocular, actuant com a lupa, augmenta un segon cop la imatge. Atès que la distància del punt proper de l'ull humà és aproximadament $\SI{25}{\centi\metre}$, s'ha d'escollir un ocular de focal inferior a aquesta xifra, ja que l'augment angular de la imatge ve donat per
\begin{equation}\label{eq:augmentull}
  \Gamma'_\textup{ocu} = \frac{\SI{25}{\centi\metre}}{f'_\textup{ocu}}
\end{equation}

\begin{figure}[H]
\centering
  \includegraphics[scale=0.8]{Microscopi.PNG}
  \captionof{figure}{Traçat de raigs per al microscopi compost.}
\label{fig:microscopi}
\end{figure}


Un cop dissenyat el sistema d'il·luminació, s'ha col·locat l'objecte a observar a poca distància de la lent col·limadora (per tal d'estalviar espai en en el banc) i s'ha procedit a muntar les lents del microscopi. Com a objectiu s'ha escollit una lent de potència $P_\textup{obj}^\textup{exp} = \SI{9.5 \pm 0.3}{\dioptre}$ i diàmetre $\diameter_\textup{obj} = \SI{5.5 \pm 0.1}{\centi\metre}$ i per a l'ocular una altra lent de potència $P_\textup{ocu}^\textup{exp} = \SI{20.5 \pm 0.3}{\dioptre}$ i diàmetre $\diameter_\textup{ocu} = \SI{5.0 \pm 0.1}{\centi\metre}$. Cal destacar que les anteriors focals són les mesurades amb el frontofocòmetre, fet que, donada la seva precisió limitada, introdueix notables discrepàncies en els càlculs posteriors. Per a obtenir mesures més compatibles, s'han considerat les potències nominals de les lents, $P_\textup{obj} = \SI{10.0 \pm 0.1}{\dioptre}$ i $P_\textup{ocu} = \SI{20.0 \pm 0.1}{\dioptre}$. Així doncs, aplicant \eqref{eq:focals} obtenim que les focals de les lents són $f'_\textup{obj} = \SI{10.0 \pm 0.1}{\centi\metre}$ i $f'_\textup{ocu} = \SI{5.00 \pm 0.03}{\centi\metre}$.

L'objectiu, que treballa com un projector, s'ha col·locat davant de la diapositiva a una distància superior a la seva focal. Mitjançant una pantalla, s'ha trobat la posició en què s'obté una imatge prou augmentada de la diapositiva a través de l'objectiu, que ha de correspondre a la posició del pla objecte de l'ocular per tal que els raigs surtin paral·lels i l'ull els pugui acomodar. Les distàncies entre els elements òptics mesurades amb una cinta mètrica són
\begin{itemize}[label=--]
  \item la distància entre l'objectiu i l'ocular és $d_{\textup{obj} \rightarrow \textup{ocu}}^\textup{exp} = \SI{82.3 \pm 0.1}{\centi\metre}$,
  %\item la distància entre el col·limador i l'objectiu és $d_{\textup{col} \rightarrow \textup{obj}} = \SI{17.3 \pm 0.1}{\centi\metre}$ i
  \item la distància entre l'objectiu i la diapositiva $d_{\textup{obj} \rightarrow \textup{dia}}^\textup{exp} = -s_\textup{obj} =  \SI{11.3 \pm 0.1}{\centi\metre}$.
\end{itemize}

Aplicant l'equació de les lents, podem calcular la distància entre l'objectiu i l'ocular donades les seves focals. Fem servir que
\begin{equation}\label{eq:lents}
  -\frac{1}{s_\textup{obj}} + \frac{1}{s'_\textup{obj}} = \frac{1}{f'_\textup{obj}} \Rightarrow s'_\textup{obj} = \frac{s_\textup{obj}\, f'_\textup{obj}}{s_\textup{obj} + f'_\textup{obj}}
\end{equation}
i obtenim que $s'_\textup{obj} = \SI{87 \pm 10}{\centi\metre}$. Així doncs, la distància entre l'objectiu i l'ocular ve donada per $d_{\textup{obj} \rightarrow \textup{ocu}} = s'_\textup{obj} + f'_\textup{ocu} = \SI{92 \pm 10}{\centi\metre}$, valor compatible amb l'obtingut amb la cinta mètrica.

Així doncs, tenim que la longitud òptica del tub, que no és més que la distància entre el focus imatge de l'objectiu i el focus objecte de l'ocular, vindrà donada per
\begin{equation}\label{eq:longtub}
  t = d_{\textup{obj} \rightarrow \textup{ocu}} - \left( f'_\textup{obj} + f'_\textup{ocu}\right)
\end{equation}
i val $t = \SI{77 \pm 10}{\centi\metre}$.

Ara, ens proposem calcular l'augment angular del microscopi $\Gamma'$. Per fer-ho, primer calculem l'augment lateral de l'objectiu, segons
\begin{equation}\label{eq:augobj}
  \beta'_\textup{obj} = -\frac{t}{f'_\textup{obj}}%, \hspace{3mm} \delta \beta'_\textup{obj} = \sqrt{\left(\frac{\delta t}{f'_\textup{obj}}\right)^2 + \left[\frac{t}{\left(f'_\textup{obj}\right)^2} \, \delta f'_\textup{obj} \right]^2}
\end{equation}
i obtenim que $\beta'_\textup{obj} = -7.7 \pm 1.0$. Així mateix, calculem l'augment angular de l'ocular segons \eqref{eq:augmentull} i obtenim $\Gamma'_\textup{ocu} = 5.00 \pm 0.03$. Conegudes aquestes dades, podem expressar l'augment angular total del sistema com
\begin{equation}\label{eq:augmentotal}
  \Gamma' = \beta'_\textup{obj} \, \Gamma'_\textup{ocu}%, \hspace{3mm} \delta\Gamma' = \sqrt{\left(\Gamma'_\textup{ocu}\delta\beta'_\textup{obj}\right)^2 + \left(\beta'_\textup{obj}\delta\Gamma'_\textup{ocu}\right)^2}
\end{equation}
que resulta $\Gamma' = -39 \pm 5$.

L'emergència de la pupil·la de sortida vindrà donada per l'equació de les lents segons
\begin{equation} \label{eq:emergenciamicro}
  \frac{1}{d_{\textup{obj} \rightarrow \textup{ocu}}} + \frac{1}{e} = \frac{1}{f'_\textup{ocu}} \Rightarrow e = \frac{f'_\textup{ocu} \, d_{\textup{obj} \rightarrow \textup{ocu}}}{f'_\textup{ocu} + d_{\textup{obj} \rightarrow \textup{ocu}}}
\end{equation}
i val $e = \SI{4.74 \pm 0.04}{\centi\metre}$.

Cal remarcar que s'ha observat una notable aberració esfèrica en els punts de l'espai imatge més allunyats de l'eix òptic. Així mateix, la coloració de la imatge resultava estranya, fet que s'atribueix a la font lumínica.

Amb les dades donades, es pot estimar el valor real de la potència de l'objectiu, coneguda la focal de l'ocular i la distància entre les lents. Emprant l'equació de les lents, veiem que
\begin{equation}\label{eq:estfocal}
  P_\textup{obj}^\textup{teo} = -\frac{1}{s_\textup{obj}} + \frac{1}{d_{\textup{obj} \rightarrow \textup{ocu}}^\textup{exp} - f'_\textup{ocu}}
\end{equation}
i obtenim que $P_\textup{obj}^\textup{teo} = \SI{10.00 \pm 0.08}{\dioptre}$, valor compatible tant amb la potència mesurada amb el frontofocòmetre com la nominal.

Finalment, s'ha repetit el muntatge usant un altre ocular de potència $\tilde{P}_\textup{ocu}^\textup{exp} = \SI{9.5 \pm 0.3}{\dioptre}$ segons el frontofocòmetre i potència nominal $P_\textup{ocu} = \SI{10.0 \pm 0.1}{\dioptre}$, de diàmetre $\tilde{\diameter}_\textup{ocu} = \SI{5.5 \pm 0.1}{\centi\metre}$. La focal nominal calculada segons \eqref{eq:focals} resulta $\tilde{f'}_\textup{ocu} = \SI{10.0 \pm 0.1}{\centi\metre}$. La distància entre l'objectiu i l'ocular mesurat amb una cinta mètrica és de $\tilde{d}_{\textup{obj} \rightarrow \textup{ocu}}^\textup{exp} = \SI{90.6 \pm 0.1}{\centi\metre}$.

En aquest cas obtenim, doncs, que  $\tilde{d}_{\textup{obj} \rightarrow \textup{ocu}} = \tilde{s'}_\textup{obj} + \tilde{f'}_\textup{ocu} = \SI{97 \pm 10}{\centi\metre}$, valor que concorda amb el mesurat amb la cinta mètrica. La longitud òptica del tub és $\tilde{t} = \SI{77 \pm 10}{\centi\metre}$, l'augment lateral de l'objectiu es manté constant, l'augment angular de l'ocular és $\tilde{\Gamma}'_\textup{ocu} = 2.50 \pm 0.02$, d'on l'augment angular total del sistema és $\tilde{\Gamma}' = -19.3 \pm 0.3$ i l'emergència de la pupil·la de sortida $\tilde{e} = \SI{9.07 \pm 0.12}{\centi\metre}$.

Notem que ara s'obté aproximadament la meitat d'augments que en el cas anterior, a canvi de tenir menys aberració esfèrica.

En tots els casos, la col·locació d'un diafragma iris en el punt focal objecte de l'ocular limita el camp del sistema.




\appendix
\newpage
% \section{Fórmules estadístiques}\label{sec:apendix}
\setcounter{equation}{0}
\renewcommand{\theequation}{\thesection.\arabic{equation}}
%\renewcommand{\thesubsection}{\thesection.\Roman{subsection}}
\subsection{Paràmetres centrals i de dispersió}
Donada una mostra de $N$ elements $\left\{x_1,\dots,x_N\right\}$, definim els següents paràmetres estadístics.
\begin{itemize}[label=--]
    \item Mitjana aritmètica \begin{equation}\label{est:mitjana}
            \langle x\rangle =\frac{1}{N} \sum_{i=1}^{N} x_i
        \end{equation}
    \item Desviació estàndard\footnote{S'ha fet servir la correcció de Bessel $\sqrt{N/(N-1)}$ de la desviació estàndard poblacional.} \begin{equation}\label{est:stdev}
            {\displaystyle \sigma_x ={\sqrt {{\frac {1}{N-1}}\sum _{i=1}^{N}(x_{i}-\langle x \rangle )^{2}}}}
        \end{equation}
    \item Error estàndard \begin{equation}\label{est:err}
   {\displaystyle \delta{x} ={\frac {\sigma_x}{\sqrt {N}}}}
\end{equation}

\end{itemize}
\subsection{Estimacions lineals}
Per al càlcul de les estimacions lineals s'usa la funció \texttt{ESTIMACION.LINEAL} del full de càlcul \texttt{Microsoft Excel}, que proporciona el pendent i l'ordenada a l'origen de la recta de regressió amb les seves corresponents incerteses, així com el coeficient de correlació $R^2$ i l'error estàndard de la regressió.
% El funcionament complet de la funció es pot consultar a \url{https://support.office.com/es-es/article/estimacion-lineal-funci\%C3\%B3n-estimacion-lineal-84d7d0d9-6e50-4101-977a-fa7abf772b6d}.

\subsection{Test \texorpdfstring{$\chi^2$}{x2}}
Donat un ajust lineal $y=A\,x+B$, amb incertesa en la variable dependent $\delta y$ i error estàndard de la regressió $\delta y_\textup{reg}$, es defineix el \emph{coeficient $\chi^2$} com
\begin{equation}
    \chi^2 = \nu\, \left(\frac{\delta y_\textup{reg}}{\delta y}\right)
\end{equation}
on $\nu$ es el nombre de graus de llibertat de l'ajust. Amb això, podem definir el \emph{coeficient reduït $\chi^2_\nu$} com
\begin{equation}
    \chi^2_\nu = \frac{\delta y_\textup{reg}}{\delta y}
\end{equation}
El valor d'aquest paràmetre ens indica la bondat de l'ajust realitzat. Tenim que
\begin{enumerate}[label=(\alph*)]
    \item si \underline{$\delta y_\textup{reg} \ll \delta y$} o \underline{$\chi^2_\nu \ll 1$}, l'ajust és acceptable i probablement s'hagi sobreestimat $\delta y$;
    \item si \underline{$\delta y_\textup{reg} \lesssim \delta y$} o \underline{$\chi^2_\nu \lesssim 1$}, l'ajust és acceptable; i
    \item si \underline{$\delta y_\textup{reg} \gg \delta y$} o \underline{$\chi^2_\nu \gg 1$} l'ajust no és acceptable.
\end{enumerate}

\end{document}
