\documentclass[11pt,a4paper,catalan]{article}
\author{Marc Ballestero Ribó}

%___PAQUETS NECESSARIS___%
\usepackage[utf8]{inputenc}
\usepackage{amsthm,amssymb,amsmath,mathrsfs}
\usepackage[catalan]{babel}
\usepackage{wasysym}
\usepackage[margin=25mm]{geometry}
\usepackage[bottom]{footmisc}
\usepackage{enumerate}
\usepackage{array, boldline, makecell, booktabs, dcolumn}
\usepackage[svgnames, table]{xcolor}
\usepackage[all]{xy}
\usepackage{parskip}
\usepackage{float}
\usepackage{fancyhdr}
\usepackage{graphicx}
\usepackage{multirow,multicol}
\usepackage{pgfplots, tikz}
\usepackage{pgfplotstable}
\usepackage{graphics}
\usepackage{arydshln}
\usepackage{caption}
\usepackage{siunitx}
\usepackage{chngpage}
\usepackage{enumitem}
\usepackage{hyperref}
\hypersetup{
    colorlinks=true,
    linkcolor=black,
    filecolor=black,
    urlcolor=blue,
}

\sisetup{separate-uncertainty=true, exponent-product=\cdot}
\DeclareSIUnit\dioptre{D}
\DeclareSIUnit\arbitrary{u.a.}
\decimalpoint



%\sisetup{locale = FR}

%___NOUS TIPUS DE COLUMNES PER LES TAULES___%
\newcolumntype{L}[1]{>{\raggedright\let\newline\\\arraybackslash\hspace{0pt}}m{#1}}
\newcolumntype{C}[1]{>{\centering\let\newline\\\arraybackslash\hspace{0pt}}m{#1}}
\newcolumntype{R}[1]{>{\raggedleft\let\newline\\\arraybackslash\hspace{0pt}}m{#1}}
\newcolumntype{d}[1]{D{.}{.}{#1}}

%___COMANDAMENTS PERSONALITZATS___%
\providecommand{\keywords}[1]{\textsc{{Paraules clau---}} #1}
\newcommand{\diff}{\mathrm{d}}
\newcommand{\nind}{\noindent}
\newcommand{\res}[3]{$\left ( #1 \pm #2 \right ) \, \si{#3}$}
\newcommand{\figref}[1]{figura \ref{#1}}
\newcommand{\tabref}[1]{taula \ref{#1}}
\renewcommand\thesection{\arabic{section}}
\newcommand{\data}{\today}


%___CAPÇALERA I PEU DE PÀGINA___%
\pagestyle{fancy}
\headheight=16pt
\setlength{\headsep}{1cm}
\fancypagestyle{headings}{
\cfoot{\thepage}}
\lhead{\scshape Òptica \textup{2020-21}}
\rhead{\data \\ Marc Ballestero Ribó}

%___CONFIGURACIÓ DEL PGFPLOTS___%
\pgfplotsset{compat = 1.15}
\pgfplotsset{
    legend image with text/.style={
        legend image code/.code={%
            \node[anchor=center] at (0.3cm,0cm) {#1};
        }
    },
    log x ticks with fixed point/.style={
      xticklabel={
        \pgfkeys{/pgf/fpu=true}
        \pgfmathparse{exp(\tick)}%
        \pgfmathprintnumber[fixed relative, precision=0]{\pgfmathresult}
        \pgfkeys{/pgf/fpu=false}
      }
    },
    log y ticks with fixed point/.style={
      yticklabel={
        \pgfkeys{/pgf/fpu=true}
        \pgfmathparse{exp(\tick)}%
        \pgfmathprintnumber[fixed zerofill, precision=0]{\pgfmathresult}
        \pgfkeys{/pgf/fpu=false}
      }
  },
}
\usetikzlibrary{spy}

%___CONFIGURACIÓ DELS PEUS DE FIGURA___%
\captionsetup[figure]{justification = centering,labelfont={small,sc},textfont=small}
\captionsetup[table]{name = Taula, justification = centering,labelfont={small,sc},textfont=small}

%___DIRECTORI D'IMATGES___%
\graphicspath{ {./Imatges/} }




%___TÍTOL___%

\title{\vspace{-1.5cm} Òptica \\ \vspace{1mm} Práctica 2. Polarització de la llum}
\author{Marc Ballestero Ribó - Grup D2}
\date{\data}


\begin{document}
\maketitle

\ifx
 \begin{abstract}
  \nind El objetivo de esta práctica es estudiar las oscilaciones amortiguadas y forzadas de un péndulo de Pohl. Primeramente, se ha estudiado el comportamiento del péndulo cuando se le aplica una fuerza de fricción dependiente de su velocidad de rotación, dada por un freno magnético. En este caso se ha obtenido que el péndulo se comporta como un oscilador armónico débilmente amortiguado, con constante de amortiguación $\beta = \SI{0.177\pm0.007}{\per\second}$. Una vez conocido el comportamiento del péndulo ante la fuerza dada por el freno magnético, se ha aplicado un forzamiento periódico al sistema, dado por un motor de potencia regulable. Estudiando el comportamiento estacionario del péndulo para distintas frecuencias de forzamiento, se ha conseguido aproximar el valor de la frecuencia de resonancia en amplitud del sistema, dado por $\omega_\textup{R} =\SI{2.855  \pm 0.003}{\per\second}$.

 \end{abstract}
\fi

\section{Polarització lineal} \label{sec:polLineal}

En aquest apartat ens proposem estudiar el comportament d'un feix de llum natural quan passa per un sistema de dos polaritzadors lineals consecutius i enfrontats. Així doncs, s'ha col·locat, davant de la font de llum LED, un primer polaritzador lineal al qual ens referirem com a \emph{polaritzador} i tot seguit un altre dispositiu similar que anomenarem \emph{analitzador}, tot procurant que en els estats de màxima transmissió no se saturi el díode detector col·locat al final de la línia de llum. Aquest element és el que, connectat a un amperímetre en escala de \si{\micro\ampere}, ens informa sobre la intensitat de la llum un cop ha travessat el sistema òptic. Com que la resposta del díode a la il·luminació és lineal, considerarem que la intensitat mesurada per l'amperímetre ens dóna la mesura de la intensitat lumínica en unitats arbitràties (\textup{u.a.}).

Fixat el polaritador, s'han près mesures d'intensitat modificant l'angle de l'eix de l'analitzador $\beta$ en intervals de $\SI{5 \pm 1}{\degree}$, els punts experimentals associats a les quals es representen en la figura \ref{fig:Lineal_I(b)}.

\input{Figures/Lineal_I(b)}

La llei de Malus relaciona, en un sistema com aquest, la intensitat detectada vers l'angle relatiu entre els eixos dels polaritzadors, segons
\begin{equation} \label{eq:malus}
 I = I_0 \, \cos^2{\alpha}
\end{equation}
on $I_0$ és la intensitat abans de transmetre's pels polaritzadors i $\alpha$ l'angle relatiu entre els eixos del polaritzador i l'analitzador.

Per tal d'establir l'origen de fases i el paràmetre d'intensitat màxima, s'han mesurat els punts on el valor marcat pel díode és extrem\footnote{Malgrat es podrien haver fet servir els valors obtinguts prèviament, s'ha repetit la mesura per evitar l'error de discretització associat a prendre dades en intervals de \SI{5 \pm 1}{\degree}.}, obtenint que
\begin{align*}
 \left[\beta_\textup{max}, I_\textup{max}\right] & = \left[\SI{54\pm 1}{\degree}, \SI{7.7 \pm 0.1}{\arbitrary}\right]   \\
 \left[\beta_\textup{min}, I_\textup{min}\right] & = \left[\SI{144\pm 1}{\degree}, \SI{0.0 \pm 0.1}{\arbitrary}\right]. \\
\end{align*}
Notem que es compleix que $\left | \beta_\textup{max} - \beta_\textup{min} \right | = \SI{90 \pm 1}{\degree}$, la qual cosa ens indica que les mesures realitzades s'ajusten a la predicció teòrica.

Prendrem, doncs, com a origen de fases l'angle per al qual la intensitat és máxima, amb la qual cosa tindrem que $\left[\alpha_0, I_0\right] = \left[\SI{54\pm 1}{\degree}, \SI{7.7 \pm 0.1}{\arbitrary}\right]$. Amb això, podem calcular l'angle relatiu entre el polaritzador i l'analitzador com
\begin{equation} \label{eq:transformaAngle}
 \alpha := \beta - \alpha_0 .
\end{equation}
Amb aquesta relació s'ha ajustat la corba que se superposa als punts de la figura \ref{fig:Lineal_I(b)}, que, vist que talla tots els punts experimentals i les corresponents barres d'error, ens indica que les dades preses s'ajusten al model teòric que prediu \eqref{eq:malus}.

\input{Figures/Lineal_Malus}

Encara més, s'han representat els punts $I\left(\cos^2{\alpha}\right)$ i s'han ajustat a una recta de regressió de la forma $I = I_0'\, \cos^2{\alpha} + I_\textup{r}$, tot representat a la figura \ref{fig:Lineal_Malus}. Els paràmetres obtinguts es presenten a la taula \ref{tab:Lineal_Ajust}.

\begin{table}[H]
\centering
    \begin{tabular}{l | r}
        \hline
        $I_0'$ (\si{\arbitrary}) & $7.66 \pm 0.04$ \\
        $I_\textup{r}$ (\si{\arbitrary})& $0.0 \pm 0.3$ \\
        $R^2$ & $0.998$ \\
        $\left(\delta I \right)_\textup{reg}$ (\si{\arbitrary}) & $0.13$ \\
        \hline
    \end{tabular}
\captionof{table}{Paràmetres de l'ajust lineal $I = I_0' \, \cos^{\alpha} + I_\textup{r}$, amb les corresponents incerteses.}
\label{tab:Lineal_Ajust}
\end{table}


Primerament, observem que els valors de la intensitat màxima obtinguts experimentalment i a partir de l'ajust satisfan que
\begin{equation*}
   \SI{0.04}{\arbirtary} = \left |I_0 - I_0' \right | \leq \sqrt{\left(\delta I_0\right)^2 + \left(\delta I_0'\right)^2} = \SI{0.11}{\arbitrary}
\end{equation*}
i, per tant, són compatibles.

D'altra banda, l'ordenada a l'origen de l'ajust $I_\textup{r}$ ens informa de la intensistat residual detectada pel díode, que correspon a la radiació lumínica de fons del laboratori, que, malgrat prendre's les dades en condicions de foscor, no es pot anul·lar totalment. Tot i així, notem que el valor de $I_\textup{r}$ és clarament compatible amb zero.

\appendix
\newpage
% \section{Fórmules estadístiques}\label{sec:apendix}
\setcounter{equation}{0}
\renewcommand{\theequation}{\thesection.\arabic{equation}}
%\renewcommand{\thesubsection}{\thesection.\Roman{subsection}}
\subsection{Paràmetres centrals i de dispersió}
Donada una mostra de $N$ elements $\left\{x_1,\dots,x_N\right\}$, definim els següents paràmetres estadístics.
\begin{itemize}[label=--]
    \item Mitjana aritmètica \begin{equation}\label{est:mitjana}
            \langle x\rangle =\frac{1}{N} \sum_{i=1}^{N} x_i
        \end{equation}
    \item Desviació estàndard\footnote{S'ha fet servir la correcció de Bessel $\sqrt{N/(N-1)}$ de la desviació estàndard poblacional.} \begin{equation}\label{est:stdev}
            {\displaystyle \sigma_x ={\sqrt {{\frac {1}{N-1}}\sum _{i=1}^{N}(x_{i}-\langle x \rangle )^{2}}}}
        \end{equation}
    \item Error estàndard \begin{equation}\label{est:err}
   {\displaystyle \delta{x} ={\frac {\sigma_x}{\sqrt {N}}}}
\end{equation}

\end{itemize}
\subsection{Estimacions lineals}
Per al càlcul de les estimacions lineals s'usa la funció \texttt{ESTIMACION.LINEAL} del full de càlcul \texttt{Microsoft Excel}, que proporciona el pendent i l'ordenada a l'origen de la recta de regressió amb les seves corresponents incerteses, així com el coeficient de correlació $R^2$ i l'error estàndard de la regressió.
% El funcionament complet de la funció es pot consultar a \url{https://support.office.com/es-es/article/estimacion-lineal-funci\%C3\%B3n-estimacion-lineal-84d7d0d9-6e50-4101-977a-fa7abf772b6d}.

\subsection{Test \texorpdfstring{$\chi^2$}{x2}}
Donat un ajust lineal $y=A\,x+B$, amb incertesa en la variable dependent $\delta y$ i error estàndard de la regressió $\delta y_\textup{reg}$, es defineix el \emph{coeficient $\chi^2$} com
\begin{equation}
    \chi^2 = \nu\, \left(\frac{\delta y_\textup{reg}}{\delta y}\right)
\end{equation}
on $\nu$ es el nombre de graus de llibertat de l'ajust. Amb això, podem definir el \emph{coeficient reduït $\chi^2_\nu$} com
\begin{equation}
    \chi^2_\nu = \frac{\delta y_\textup{reg}}{\delta y}
\end{equation}
El valor d'aquest paràmetre ens indica la bondat de l'ajust realitzat. Tenim que
\begin{enumerate}[label=(\alph*)]
    \item si \underline{$\delta y_\textup{reg} \ll \delta y$} o \underline{$\chi^2_\nu \ll 1$}, l'ajust és acceptable i probablement s'hagi sobreestimat $\delta y$;
    \item si \underline{$\delta y_\textup{reg} \lesssim \delta y$} o \underline{$\chi^2_\nu \lesssim 1$}, l'ajust és acceptable; i
    \item si \underline{$\delta y_\textup{reg} \gg \delta y$} o \underline{$\chi^2_\nu \gg 1$} l'ajust no és acceptable.
\end{enumerate}

\end{document}
