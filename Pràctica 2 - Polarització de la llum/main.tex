\documentclass[11pt,a4paper,catalan]{article}
\author{Marc Ballestero Ribó}

%___PAQUETS NECESSARIS___%
\usepackage[utf8]{inputenc}
\usepackage{amsthm,amssymb,amsmath,mathrsfs}
\usepackage[catalan]{babel}
\usepackage{wasysym}
\usepackage[margin=25mm]{geometry}
\usepackage[bottom]{footmisc}
\usepackage{enumerate}
\usepackage{array, boldline, makecell, booktabs, dcolumn}
\usepackage[svgnames, table]{xcolor}
\usepackage[all]{xy}
\usepackage{parskip}
\usepackage{float}
\usepackage{fancyhdr}
\usepackage{graphicx}
\usepackage{multirow,multicol}
\usepackage{pgfplots, tikz}
\usepackage{pgfplotstable}
\usepackage{graphics}
\usepackage{arydshln}
\usepackage{caption}
\usepackage{siunitx}
\usepackage{chngpage}
\usepackage{enumitem}
\usepackage{hyperref}
\hypersetup{
    colorlinks=true,
    linkcolor=black,
    filecolor=black,
    urlcolor=blue,
}

\sisetup{separate-uncertainty=true, exponent-product=\cdot}
\DeclareSIUnit\dioptre{D}
\DeclareSIUnit\arbitrary{u.a.}
\decimalpoint



%\sisetup{locale = FR}

%___NOUS TIPUS DE COLUMNES PER LES TAULES___%
\newcolumntype{L}[1]{>{\raggedright\let\newline\\\arraybackslash\hspace{0pt}}m{#1}}
\newcolumntype{C}[1]{>{\centering\let\newline\\\arraybackslash\hspace{0pt}}m{#1}}
\newcolumntype{R}[1]{>{\raggedleft\let\newline\\\arraybackslash\hspace{0pt}}m{#1}}
\newcolumntype{d}[1]{D{.}{.}{#1}}

%___COMANDAMENTS PERSONALITZATS___%
\providecommand{\keywords}[1]{\textsc{{Paraules clau---}} #1}
\newcommand{\diff}{\mathrm{d}}
\newcommand{\nind}{\noindent}
\newcommand{\res}[3]{$\left ( #1 \pm #2 \right ) \, \si{#3}$}
\newcommand{\figref}[1]{figura \ref{#1}}
\newcommand{\tabref}[1]{taula \ref{#1}}
\renewcommand\thesection{\arabic{section}}
\newcommand{\data}{\today}


%___CAPÇALERA I PEU DE PÀGINA___%
\pagestyle{fancy}
\headheight=16pt
\setlength{\headsep}{1cm}
\fancypagestyle{headings}{
\cfoot{\thepage}}
\lhead{\scshape Òptica \textup{2020-21}}
\rhead{\data \\ Marc Ballestero Ribó}

%___CONFIGURACIÓ DEL PGFPLOTS___%
\pgfplotsset{compat = 1.15}
\pgfplotsset{
    legend image with text/.style={
        legend image code/.code={%
            \node[anchor=center] at (0.3cm,0cm) {#1};
        }
    },
    log x ticks with fixed point/.style={
      xticklabel={
        \pgfkeys{/pgf/fpu=true}
        \pgfmathparse{exp(\tick)}%
        \pgfmathprintnumber[fixed relative, precision=0]{\pgfmathresult}
        \pgfkeys{/pgf/fpu=false}
      }
    },
    log y ticks with fixed point/.style={
      yticklabel={
        \pgfkeys{/pgf/fpu=true}
        \pgfmathparse{exp(\tick)}%
        \pgfmathprintnumber[fixed zerofill, precision=0]{\pgfmathresult}
        \pgfkeys{/pgf/fpu=false}
      }
  },
}
\usetikzlibrary{spy}

%___CONFIGURACIÓ DELS PEUS DE FIGURA___%
\captionsetup[figure]{justification = justified,labelfont={small,sc},textfont=small}
\captionsetup[table]{name = Taula, justification = justified,labelfont={small,sc},textfont=small}

%___DIRECTORI D'IMATGES___%
\graphicspath{ {./Imatges/} }




%___TÍTOL___%

\title{\vspace{-1.5cm} Òptica \\ \vspace{1mm} Práctica 2. Polarització de la llum}
\author{Marc Ballestero Ribó - Grup D2}
\date{\data}


\begin{document}
\maketitle

\ifx
 \begin{abstract}
  \nind El objetivo de esta práctica es estudiar las oscilaciones amortiguadas y forzadas de un péndulo de Pohl. Primeramente, se ha estudiado el comportamiento del péndulo cuando se le aplica una fuerza de fricción dependiente de su velocidad de rotación, dada por un freno magnético. En este caso se ha obtenido que el péndulo se comporta como un oscilador armónico débilmente amortiguado, con constante de amortiguación $\beta = \SI{0.177\pm0.007}{\per\second}$. Una vez conocido el comportamiento del péndulo ante la fuerza dada por el freno magnético, se ha aplicado un forzamiento periódico al sistema, dado por un motor de potencia regulable. Estudiando el comportamiento estacionario del péndulo para distintas frecuencias de forzamiento, se ha conseguido aproximar el valor de la frecuencia de resonancia en amplitud del sistema, dado por $\omega_\textup{R} =\SI{2.855  \pm 0.003}{\per\second}$.

 \end{abstract}
\fi

\section{Polarització lineal} \label{sec:polLineal}

En aquest apartat ens proposem estudiar el comportament d'un feix de llum natural quan passa per un sistema de dos polaritzadors lineals consecutius i enfrontats. Així doncs, s'ha col·locat, davant de la font de llum LED, un primer polaritzador lineal al qual ens referirem com a \emph{polaritzador} i tot seguit un altre dispositiu similar que anomenarem \emph{analitzador}, tot procurant que en els estats de màxima transmissió no se saturi el díode detector col·locat al final de la línia de llum. Aquest element és el que, connectat a un amperímetre en escala de \si{\micro\ampere}, ens informa sobre la intensitat de la llum un cop ha travessat el sistema òptic. Com que la resposta del díode a la il·luminació és lineal, considerarem que la intensitat mesurada per l'amperímetre ens dóna la mesura de la intensitat lumínica en unitats arbitràties (\textup{u.a.}).

Fixat el polaritador, s'han près mesures d'intensitat modificant l'angle de l'eix de l'analitzador $\beta$ en intervals de $\SI{5 \pm 1}{\degree}$, els punts experimentals associats a les quals es representen en la figura \ref{fig:Lineal_I(b)}.

\input{Figures/Lineal_I(b)}

La llei de Malus relaciona, en un sistema com aquest, la intensitat detectada vers l'angle relatiu entre els eixos dels polaritzadors, segons
\begin{equation} \label{eq:malus}
 I = I_0 \, \cos^2{\alpha}
\end{equation}
on $I_0$ és la intensitat abans de transmetre's pels polaritzadors i $\alpha$ l'angle relatiu entre els eixos del polaritzador i l'analitzador.

Per tal d'establir l'origen de fases i el paràmetre d'intensitat màxima, s'han mesurat els punts on el valor marcat pel díode és extrem\footnote{Malgrat es podrien haver fet servir els valors obtinguts prèviament, s'ha repetit la mesura per evitar l'error de discretització associat a prendre dades en intervals de \SI{5 \pm 1}{\degree}.}, obtenint que
\begin{align*}
 \left[\beta_\textup{max}, I_\textup{max}\right] & = \left[\SI{54\pm 1}{\degree}, \SI{7.7 \pm 0.1}{\arbitrary}\right]   \\
 \left[\beta_\textup{min}, I_\textup{min}\right] & = \left[\SI{144\pm 1}{\degree}, \SI{0.0 \pm 0.1}{\arbitrary}\right]. \\
\end{align*}
Notem que es compleix que $\left | \beta_\textup{max} - \beta_\textup{min} \right | = \SI{90 \pm 1}{\degree}$, la qual cosa ens indica que les mesures realitzades s'ajusten a la predicció teòrica.

Prendrem, doncs, com a origen de fases l'angle per al qual la intensitat és máxima, amb la qual cosa tindrem que $\left[\alpha_0, I_0\right] = \left[\SI{54\pm 1}{\degree}, \SI{7.7 \pm 0.1}{\arbitrary}\right]$. Amb això, podem calcular l'angle relatiu entre el polaritzador i l'analitzador com
\begin{equation} \label{eq:transformaAngle}
 \alpha := \beta - \alpha_0 .
\end{equation}
Amb aquesta relació s'ha ajustat la corba que se superposa als punts de la figura \ref{fig:Lineal_I(b)}, que, vist que talla tots els punts experimentals i les corresponents barres d'error, ens indica que les dades preses s'ajusten al model teòric que prediu \eqref{eq:malus}.

\input{Figures/Lineal_Malus}

Encara més, s'han representat els punts $I\left(\cos^2{\alpha}\right)$ i s'han ajustat a una recta de regressió de la forma $I = I_0'\, \cos^2{\alpha} + I_\textup{r}$, tot representat a la figura \ref{fig:Lineal_Malus}. Els paràmetres obtinguts es presenten a la taula \ref{tab:Lineal_Ajust}.

\begin{table}[H]
\centering
    \begin{tabular}{l | r}
        \hline
        $I_0'$ (\si{\arbitrary}) & $7.66 \pm 0.04$ \\
        $I_\textup{r}$ (\si{\arbitrary})& $0.0 \pm 0.3$ \\
        $R^2$ & $0.998$ \\
        $\left(\delta I \right)_\textup{reg}$ (\si{\arbitrary}) & $0.13$ \\
        \hline
    \end{tabular}
\captionof{table}{Paràmetres de l'ajust lineal $I = I_0' \, \cos^{\alpha} + I_\textup{r}$, amb les corresponents incerteses.}
\label{tab:Lineal_Ajust}
\end{table}


Primerament, observem que els valors de la intensitat màxima obtinguts experimentalment i a partir de l'ajust satisfan que
\begin{equation*}
 \SI{0.04}{\arbitrary} = \left |I_0 - I_0' \right | \leq \sqrt{\left(\delta I_0\right)^2 + \left(\delta I_0'\right)^2} = \SI{0.11}{\arbitrary}
\end{equation*}
i, per tant, són compatibles.

D'altra banda, l'ordenada a l'origen de l'ajust $I_\textup{r}$ ens informa de la intensistat residual detectada pel díode, que correspon a la radiació lumínica de fons del laboratori. Malgrat prendre's les dades en condicions de foscor, no es pot anul·lar totalment. Tot i així, notem que el valor de $I_\textup{r}$ és clarament compatible amb zero.

\section{Polarització el·líptica} \label{sec:polEliptica}

En aquesta secció estudiarem el comportament d'un feix de llum polaritzada el·lípticament. Per aconseguir aquest estat de polarització, s'ha fet servir el mateix sistema experimental que en l'apartat \ref{sec:polLineal}, amb una làmina retardadora de quart d'ona (també l'anomenarem \emph{làmina $\lambda/4$}) entre el polaritzador i l'analitzador. Amb això s'aconsegueix obtenir llum amb una el·lipse de polarització més o menys excèntrica segons l'angle $\varphi$ entre l'eix del polaritzador i un dels eixos principals de la làmina. Per a la recollida de dades s'ha escollit un valor de $\varphi = \SI{20 \pm 1}{\degree}$.

S'han près valors de la intensitat marcada per l'amperímetre en intervals de $\SI{5 \pm 1}{\degree}$ per a l'angle de l'escala de l'analitzador, obtenint les dades que es representen en la figura \ref{fig:Eliptica_I(b)}.

\begin{figure}[H]
\centering
\begin{tikzpicture}[scale=0.85]
    \begin{axis}[
        width = 12cm,
        height = 7cm,
        enlarge x limits=0.10,
        enlarge y limits=0.10,
        %minor tick num = 3,
        xlabel={$\beta$ (\si{\degree})},
        ylabel={$I$ (\si{\arbitrary})},
        xtick = {0, 60,...,360},
        %scaled x tick = {real:3.141516},
        ytick={0,2,...,8},
        %xticklabels={$0$, $\frac{1}{4}\pi$, $\frac{1}{2}\pi$, $\frac{3}{4}\pi$, $\pi$, $\frac{5}{4}\pi$, $\frac{3}{2}\pi$, $\frac{7}{4}\pi$, $2\pi$},
        minor xtick={-30,-20,...,390},
        minor ytick={-1,-0.75,...,9},
        xmin = 0, xmax = 360, extra x ticks ={},
        ymin = 0, ymax = 8,
        %extra y ticks ={4.18},
        %ymode = log,
        %log y ticks with fixed point,
        legend image post style={scale=0.4},
        legend pos=south west,
        /pgf/number format/.cd,
        1000 sep={},
        %xticklabel style={xshift=-2pt},
        x tick label style={
        /pgf/number format/.cd,
        fixed,
        fixed zerofill,
        precision=0,
      },
        y tick label style={
        /pgf/number format/.cd,
        fixed,
        fixed zerofill,
        precision=0
      },
    ]
    \addplot[samples=200,domain=-5:365, color=red, thick] {5.8*(cos((x)-75))^2 + 1.2*(sin((x)-75))^2};
    %\addplot[samples=200,domain=0:8000, color=orange] {0.00606404*(x)+22.434048};
    \addlegendentry{\footnotesize $I = A_2^2 \,\cos^2{\left(\beta - \phi_0\right)} + A_1^2 \,\sin^2{\left(\beta - \phi_0\right)}$}
    %\addlegendentry{\tiny $T = 0.006064 \, Q + 22.43$, $R^2 = 0.99990$}
    %\addlegendimage{legend image with text=}
    %\addlegendentry{\footnotesize $R^2=0.9993$}
    \addplot[scatter, only marks, scatter/classes={a={mark=*, black, scale=1}, a={mark=*, black, scale=0.65}}, scatter src=explicit symbolic
        ] plot [error bars/.cd, y dir = both, y explicit]table[x = b, y = I, x error = db, y error = dI, meta=class]{Dades/Eliptica_I(b).csv};
    %\addlegendentry{\footnotesize $I(V)$}
    %\addlegendentry{\footnotesize $I_\textup{teo}(V)$}
    \end{axis}
\end{tikzpicture}
\captionof{figure}{Intensitat detectada pel díode en funció de l'angle de l'escala de l'analitzador, amb la relació teòrica deduïda de la llei de Malus superposada.}
\label{fig:Eliptica_I(b)}
\end{figure}


Sigui $\phi$ l'angle entre l'eix de l'analitzador i l'eix major de l'el·lipse de polarització. Per calcular la intensitat de la llum que arriba al diode, apliquem la llei de Malus a cadascuna de les components perpendiculars de la llum polaritzada el·lípticament, d'on
\begin{equation}\label{eq:malusEliptica}
 I\left(\phi\right) = A_2^2\, \cos^2{\phi} + A_1^2\, \sin^2{\phi} = \left(A_2^2 - A_1^2\right)\, \cos^2{\phi} + A_1^2
\end{equation}
on $A_1, A_2$ són els semieixos de l'el·lipse de polarització.

Per fixar l'origen de fases i trobar els semieixos de l'el·lipse, s'han mesurat els punts on la intensitat és extema, obtenint
\begin{align*}
 \left[\beta_\textup{max}, I_\textup{max}\right] & = \left[\SI{75\pm 1}{\degree}, \SI{5.8 \pm 0.1}{\arbitrary}\right]   \\
 \left[\beta_\textup{min}, I_\textup{min}\right] & = \left[\SI{165\pm 1}{\degree}, \SI{1.2 \pm 0.1}{\arbitrary}\right]. \\
\end{align*}

Així doncs, l'origen de fases correspon al punt pel qual la intensitat és màxima, $\phi_0 = \SI{75 \pm 1}{\degree}$, i introduint a \eqref{eq:malusEliptica} els valors mesurats, obtenim les longituds dels semieixos $A_1 = \SI{1.10 \pm 0.05}{\arbitrary}$ i $A_2 = \SI{2.41 \pm 0.02}{\arbitrary}$. L'angle $\phi$ l'obtindrem fent la transformació
\begin{equation} \label{eq:transformaPhi}
 \phi := \beta - \phi_0
\end{equation}
on $\beta$ és l'angle mesurat en l'escala de l'analitzador.

Amb aquests valors s'ha ajustat la corba teòrica superposada als punts de la figura \ref{fig:Eliptica_I(b)}, que, atès que talla tots els punts i barres d'error, ens indica que les dades experimentals s'ajusten al model teòric \eqref{eq:malusEliptica}.

Encara més, podem calcular l'excentricitat de l'el·lipse segons
\begin{equation} \label{eq:excentricitat}
 \varepsilon = \sqrt{1 - \frac{A_1^2}{A_2^2}}
\end{equation}
d'on s'obté que $\varepsilon = \num{0.891 \pm 0.010}$.

\begin{figure}[H]
\centering
\begin{tikzpicture}[scale=0.85]
    \begin{axis}[
        width = 12cm,
        height = 7cm,
        enlarge x limits=0.10,
        enlarge y limits=0.10,
        %minor tick num = 3,
        xlabel={$\cos^2{\left(\alpha/\si{\degree}\right)}$ ($-$)},
        ylabel={$I$ (\si{\arbitrary})},
        xtick = {0, 0.1,...,1.1},
        %scaled x tick = {real:3.141516},
        ytick={0,2,...,8},
        %xticklabels={$0$, $\frac{1}{4}\pi$, $\frac{1}{2}\pi$, $\frac{3}{4}\pi$, $\pi$, $\frac{5}{4}\pi$, $\frac{3}{2}\pi$, $\frac{7}{4}\pi$, $2\pi$},
        minor xtick={-0.1,-0.075,...,1.1},
        minor ytick={-1,-0.75,...,9},
        xmin = 0, xmax = 1, extra x ticks ={},
        ymin = 0, ymax = 8,
        %extra y ticks ={4.18},
        %ymode = log,
        %log y ticks with fixed point,
        legend image post style={scale=0.4},
        legend pos=north west,
        /pgf/number format/.cd,
        1000 sep={},
        %xticklabel style={xshift=-2pt},
        x tick label style={
        /pgf/number format/.cd,
        fixed,
        fixed zerofill,
        precision=1,
      },
        y tick label style={
        /pgf/number format/.cd,
        fixed,
        fixed zerofill,
        precision=0
      },
    ]
    \addplot[samples=200,domain=-0.05:1.05, color=red, thick] {4.55*(x)+1.22};
    %\addplot[samples=200,domain=0:8000, color=orange] {0.00606404*(x)+22.434048};
    \addlegendentry{\footnotesize $I = 4.55 \,\cos^2{\alpha} + 1.22$}
    %\addlegendentry{\tiny $T = 0.006064 \, Q + 22.43$, $R^2 = 0.99990$}
    \addlegendimage{legend image with text=}
    \addlegendentry{\footnotesize $R^2=0.996$}
    \addplot[scatter, only marks, scatter/classes={a={mark=*, black, scale=1}, a={mark=*, black, scale=0.65}}, scatter src=explicit symbolic
        ] plot [error bars/.cd, y dir = both, y explicit]table[x = cos, y = I, x error = dcos, y error = dI, meta=class]{Dades/Eliptica_Malus.csv};
    %\addlegendentry{\footnotesize $I(V)$}
    %\addlegendentry{\footnotesize $I_\textup{teo}(V)$}
    \end{axis}
\end{tikzpicture}
\captionof{figure}{Intensitat detectada pel díode en funció de cosinus al quadrat de l'angle entre l'eix de l'analitzador i l'eix major de l'el·lipse de polarització.}
\label{fig:Eliptica_Malus}
\end{figure}


Finalment, s'han representat els punts $I\left(\cos^2{\phi}\right)$ a la figura \ref{fig:Eliptica_Malus} i s'han ajustat a una recta de regressió de la forma $I = \left(A_2^2 - A_1^2\right)'\cos^2{\phi} + \left(A_1^2\right)'$, obtenint els paràmetres de la taula \ref{tab:Eliptica_Ajust}.

\begin{table}[H]
\centering
    \begin{tabular}{l | r}
        \hline
        $\left(A_2^2 - A_1^2\right)'$ (\si{\arbitrary}) & $4.55 \pm 0.04$ \\
        $\left(A_1^2\left)'$ (\si{\arbitrary})& $1.22 \pm 0.02$ \\
        $R^2$ & $0.996$ \\
        $\left(\delta I \right)_\textup{reg}$ (\si{\arbitrary}) & $0.11$ \\
        \hline
    \end{tabular}
\captionof{table}{Paràmetres de l'ajust lineal $I = \left(A_2^2 - A_1^2\right)'\cos^2{\phi} + \left(A_1^2\right)'$, amb les corresponents incerteses.}
\label{tab:Eliptica_Ajust}
\end{table}


Observem que es compleix que
\begin{align*}
 \SI{0.05}{\arbitrary} = \left |\left(A_2^2 - A_1^2\right) - \left(A_2^2 - A_1^2\right)' \right | & \leq \sqrt{\left[\delta \left(A_2^2 - A_1^2\right)\right]^2 + \left[\delta \left(A_2^2 - A_1^2\right)'\right]^2} = \SI{0.15}{\arbitrary} \\
 \SI{0.02}{\arbitrary} = \left |A_1^2 - \left(A_1^2\right)' \right |                              & \leq \sqrt{\left[\delta A_1^2\right]^2 + \left[\delta \left(A_1^2\right)'\right]^2} = \SI{0.10}{\arbitrary}
\end{align*}
i per tant els valors obtinguts mitjançant l'ajust són compatibles amb els predits amb l'equació \eqref{eq:malusEliptica}.

\section{Polartizació circular} \label{sec:polCircular}

En aquest últim apartat estudiem el comportament de la llum quan es polaritza circularment. Aquest estat de polarització s'ha aconseguit col·locant la làmina quart d'ona de manera que la direcció de polarització de la llum lineal que hi incideix formi un angle de $\SI{45}{\degree}$ amb els eixos de la làmina. Empíricament, s'ha determinat que l'angle adequat per obtenir llum polaritzada circularment és de $\SI{47\pm 1}{\degree}$ en l'escala de la làmina.

\begin{figure}[H]
\centering
\begin{tikzpicture}[scale=0.85]
    \begin{axis}[
        width = 12cm,
        height = 7cm,
        enlarge x limits=0.10,
        enlarge y limits=0.10,
        %minor tick num = 3,
        xlabel={$\beta$ (\si{\degree})},
        ylabel={$I$ (\si{\arbitrary})},
        xtick = {0, 60,...,360},
        %scaled x tick = {real:3.141516},
        ytick={0,1,...,5},
        %xticklabels={$0$, $\frac{1}{4}\pi$, $\frac{1}{2}\pi$, $\frac{3}{4}\pi$, $\pi$, $\frac{5}{4}\pi$, $\frac{3}{2}\pi$, $\frac{7}{4}\pi$, $2\pi$},
        minor xtick={-30,-20,...,390},
        minor ytick={-1,-0.75,...,7},
        xmin = 0, xmax = 360, extra x ticks ={},
        ymin = 0, ymax = 5,
        %extra y ticks ={4.18},
        %ymode = log,
        %log y ticks with fixed point,
        legend image post style={scale=0.4},
        legend pos=south west,
        /pgf/number format/.cd,
        1000 sep={},
        %xticklabel style={xshift=-2pt},
        x tick label style={
        /pgf/number format/.cd,
        fixed,
        fixed zerofill,
        precision=0,
      },
        y tick label style={
        /pgf/number format/.cd,
        fixed,
        fixed zerofill,
        precision=0
      },
    ]
    \addplot[samples=200,domain=-5:365, color=olive, very thick] {3.46};
    \addlegendentry{\footnotesize $\left \langle I_\textup{G} \right \rangle$};
    \addplot[samples=200,domain=-5:365, color=red, very thick] {1.43};
    \addlegendentry{\footnotesize $\left \langle I_\textup{R} \right \rangle$};

    \addplot[samples=200,domain=-5:365, color=olive, thick, dashed] {3.46+0.27};
    \addplot[samples=200,domain=-5:365, color=olive, thick, dashed] {3.46-0.27};

    \addplot[samples=200,domain=-5:365, color=red, thick, dashed] {1.43+0.33};
    \addplot[samples=200,domain=-5:365, color=red, thick, dashed] {1.43-0.33};
    %\addplot[samples=200,domain=0:8000, color=orange] {0.00606404*(x)+22.434048};
    %\addlegendentry{\footnotesize $I = I_0 \,\cos^2{\left(\beta - \alpha_0\right)}$}
    %\addlegendentry{\tiny $T = 0.006064 \, Q + 22.43$, $R^2 = 0.99990$}
    %\addlegendimage{legend image with text=}
    %\addlegendentry{\footnotesize $R^2=0.9993$}
    \addplot[scatter, only marks, scatter/classes={a={mark=*, olive, scale=0.65}, b={mark=*, red, scale=0.65}}, scatter src=explicit symbolic
        ] plot [error bars/.cd, y dir = both, y explicit]table[x = b, y = I, x error = db, y error = dI, meta=class]{Dades/Circular_I(b).csv};
    %\addlegendentry{\footnotesize $I(V)$}
    %\addlegendentry{\footnotesize $I_\textup{teo}(V)$}
    \end{axis}
\end{tikzpicture}
\captionof{figure}{Intensitat detectada pel díode en funció de l'angle de l'escala de l'analitzador, amb la representació de les mitjanes i desviacions estàndards corresponents.}
\label{fig:Circular_I(b)}
\end{figure}


S'han près dades de la intensitat detectada pel diode en funció de l'angle de l'escala de l'analitzador en intervals de $\SI{10 \pm 1}{\degree}$, per a llum de color verd i llum de color vermell, tot representant els punts experimentals en la figura \ref{fig:Circular_I(b)}.

Malgrat esperaríem que la llei de variació de la intensitat vers l'angle de l'analitzador fos constant, observem que els punts oscil·len lleugerament. De fet, la mitjana i la desviació estàndar de les intensitats per cada color, calculades segons \eqref{est:mitjana} i \eqref{est:stdev} respectivament, són
\begin{align*}
 \left \langle I_\textup{G} \right \rangle & = \SI{3.46 \pm 0.04}{\arbitrary} & \sigma_{I_\textup{G}} & = \SI{0.27}{\arbitrary} \\
 \left \langle I_\textup{R} \right \rangle & = \SI{1.43 \pm 0.05}{\arbitrary} & \sigma_{I_\textup{R}} & = \SI{0.33}{\arbitrary} \\
\end{align*}
\appendix
\newpage
\section{Fórmules estadístiques}\label{sec:apendix}
\setcounter{equation}{0}
\renewcommand{\theequation}{\thesection.\arabic{equation}}
%\renewcommand{\thesubsection}{\thesection.\Roman{subsection}}
\subsection{Paràmetres centrals i de dispersió}
Donada una mostra de $N$ elements $\left\{x_1,\dots,x_N\right\}$, definim els següents paràmetres estadístics.
\begin{itemize}[label=--]
    \item Mitjana aritmètica \begin{equation}\label{est:mitjana}
            \langle x\rangle =\frac{1}{N} \sum_{i=1}^{N} x_i
        \end{equation}
    \item Desviació estàndard\footnote{S'ha fet servir la correcció de Bessel $\sqrt{N/(N-1)}$ de la desviació estàndard poblacional.} \begin{equation}\label{est:stdev}
            {\displaystyle \sigma_x ={\sqrt {{\frac {1}{N-1}}\sum _{i=1}^{N}(x_{i}-\langle x \rangle )^{2}}}}
        \end{equation}
    \item Error estàndard \begin{equation}\label{est:err}
   {\displaystyle \delta{x} ={\frac {\sigma_x}{\sqrt {N}}}}
\end{equation}

\end{itemize}
\subsection{Estimacions lineals}
Per al càlcul de les estimacions lineals s'usa la funció \texttt{ESTIMACION.LINEAL} del full de càlcul \texttt{Microsoft Excel}, que proporciona el pendent i l'ordenada a l'origen de la recta de regressió amb les seves corresponents incerteses, així com el coeficient de correlació $R^2$ i l'error estàndard de la regressió.
% El funcionament complet de la funció es pot consultar a \url{https://support.office.com/es-es/article/estimacion-lineal-funci\%C3\%B3n-estimacion-lineal-84d7d0d9-6e50-4101-977a-fa7abf772b6d}.

\subsection{Test \texorpdfstring{$\chi^2$}{x2}}
Donat un ajust lineal $y=A\,x+B$, amb incertesa en la variable dependent $\delta y$ i error estàndard de la regressió $\delta y_\textup{reg}$, es defineix el \emph{coeficient $\chi^2$} com
\begin{equation}
    \chi^2 = \nu\, \left(\frac{\delta y_\textup{reg}}{\delta y}\right)
\end{equation}
on $\nu$ es el nombre de graus de llibertat de l'ajust. Amb això, podem definir el \emph{coeficient reduït $\chi^2_\nu$} com
\begin{equation}
    \chi^2_\nu = \frac{\delta y_\textup{reg}}{\delta y}
\end{equation}
El valor d'aquest paràmetre ens indica la bondat de l'ajust realitzat. Tenim que
\begin{enumerate}[label=(\alph*)]
    \item si \underline{$\delta y_\textup{reg} \ll \delta y$} o \underline{$\chi^2_\nu \ll 1$}, l'ajust és acceptable i probablement s'hagi sobreestimat $\delta y$;
    \item si \underline{$\delta y_\textup{reg} \lesssim \delta y$} o \underline{$\chi^2_\nu \lesssim 1$}, l'ajust és acceptable; i
    \item si \underline{$\delta y_\textup{reg} \gg \delta y$} o \underline{$\chi^2_\nu \gg 1$} l'ajust no és acceptable.
\end{enumerate}

\end{document}
