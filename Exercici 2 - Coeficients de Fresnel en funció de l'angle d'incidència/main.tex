\documentclass[11pt,a4paper,catalan]{article}
\author{Marc Ballestero Ribó}

%___PAQUETS NECESSARIS___%
\usepackage[utf8]{inputenc}
\usepackage{amsthm,amssymb,amsmath,mathrsfs}
\usepackage[catalan]{babel}
\usepackage{wasysym}
\usepackage[margin=25mm]{geometry}
\usepackage[bottom]{footmisc}
\usepackage{enumerate}
\usepackage{array, boldline, makecell, booktabs, dcolumn}
\usepackage[svgnames, table]{xcolor}
\usepackage[all]{xy}
\usepackage{parskip}
\usepackage{float, subfig}
\usepackage{fancyhdr}
\usepackage{graphicx}
\usepackage{multirow,multicol}
\usepackage{pgfplots, tikz}
\usepackage{pgfplotstable}
\usepackage{graphics}
\usepackage{arydshln}
\usepackage{caption}
\usepackage{siunitx}
\usepackage{chngpage}
\usepackage{enumitem}
\usepackage{hyperref}
\hypersetup{
    colorlinks=true,
    linkcolor=black,
    filecolor=black,
    urlcolor=blue,
}

\sisetup{separate-uncertainty=true, exponent-product=\cdot}
\DeclareSIUnit\dioptre{D}
\DeclareSIUnit\arbitrary{u.a.}
\DeclareSIUnit\pixel{px}
\DeclareSIUnit\bit{bit}
\DeclareSIUnit\percent{\%}
\decimalpoint



%\sisetup{locale = FR}

%___NOUS TIPUS DE COLUMNES PER LES TAULES___%
\newcolumntype{L}[1]{>{\raggedright\let\newline\\\arraybackslash\hspace{0pt}}m{#1}}
\newcolumntype{C}[1]{>{\centering\let\newline\\\arraybackslash\hspace{0pt}}m{#1}}
\newcolumntype{R}[1]{>{\raggedleft\let\newline\\\arraybackslash\hspace{0pt}}m{#1}}
\newcolumntype{d}[1]{D{.}{.}{#1}}

%___COMANDAMENTS PERSONALITZATS___%
\providecommand{\keywords}[1]{\textsc{{Paraules clau---}} #1}
\newcommand{\diff}{\mathrm{d}}
\DeclareMathOperator{\sinc}{sinc}
\DeclareMathOperator{\rect}{rect}
\DeclareMathOperator{\fcirc}{circ}
\newcommand{\nind}{\noindent}
\newcommand{\res}[3]{$\left ( #1 \pm #2 \right ) \, \si{#3}$}
\newcommand{\figref}[1]{figura \ref{#1}}
\newcommand{\tabref}[1]{taula \ref{#1}}
\renewcommand\thesection{\arabic{section}}
\newcommand{\data}{\today}


%___CAPÇALERA I PEU DE PÀGINA___%
\pagestyle{fancy}
\headheight=16pt
\setlength{\headsep}{1cm}
\fancypagestyle{headings}{
\cfoot{\thepage}}
\lhead{\scshape Òptica \textup{2020-21}}
\rhead{\data \\ Marc Ballestero Ribó}

%___CONFIGURACIÓ DEL PGFPLOTS___%
\pgfplotsset{compat = 1.15}
\pgfplotsset{
    legend image with text/.style={
        legend image code/.code={%
            \node[anchor=center] at (0.3cm,0cm) {#1};
        }
    },
    log x ticks with fixed point/.style={
      xticklabel={
        \pgfkeys{/pgf/fpu=true}
        \pgfmathparse{exp(\tick)}%
        \pgfmathprintnumber[fixed relative, precision=0]{\pgfmathresult}
        \pgfkeys{/pgf/fpu=false}
      }
    },
    log y ticks with fixed point/.style={
      yticklabel={
        \pgfkeys{/pgf/fpu=true}
        \pgfmathparse{exp(\tick)}%
        \pgfmathprintnumber[fixed zerofill, precision=0]{\pgfmathresult}
        \pgfkeys{/pgf/fpu=false}
      }
  },
}
\pgfplotsset{every tick label/.append style={font=\small}}
\usetikzlibrary{spy}

%___CONFIGURACIÓ DELS PEUS DE FIGURA___%
\captionsetup[figure]{justification = justified,labelfont={small,sc},textfont=small}
\captionsetup[table]{name = Taula, justification = justified,labelfont={small,sc},textfont=small}

%___DIRECTORI D'IMATGES___%
\graphicspath{ {./Imatges/} }




%___TÍTOL___%

\title{\vspace{-1.5cm} Òptica \\ \vspace{1mm} Exercici 2 d'avaluació continuada. Coeficients de Fresnel en funció de l'angle d'incidència}
\author{Marc Ballestero Ribó - Grup D2}
\date{\data}


\begin{document}
\maketitle

\begin{abstract}
  \nind L'objectiu d'aquest exercici és, d'una banda, representar gràficament els coeficients de Fresnel per a la interacció d'un feix de llum amb la interfase entre l'aire ($n=\num{1.000}$) i un medi dielèctric ($n'=\num{1.5263}$), i de l'altra, justificar diverses maneres d'aproximar aquests valors per a angles d'incidència petits, entre elles la consistent en imposar $r_\parallel = r_\perp$ i $t_\parallel = t_\perp$ en les reflexions de làmines plano-paral·leles.
\end{abstract}

\section*{Càlcul i representació dels coeficients de Fresnel} \label{sec:CalculRepresentacio}
Considerem un feix de llum que incideix des de l'aire ($n = \num{1.000}$) cap a un medi dielèctric d'índex de refracció $n' = \num{1.5263}$, calculat seguint les indicacions de l'enunciat.
\ifx
L'amplitud de les components dels camps incident, transmès, i reflectit, es relacionen seguint les equacions de fresnel, donades per
\begin{equation} \label{eq:fresnelt}
  t_\parallel = \frac{A'_\parallel}{A_\parallel} = \frac{2\sin{\varphi'}\cos{\varphi}}{\sin{\left(\varphi'+\varphi\right)}\cos{\left(\varphi'-\varphi\right)}}, \hspace{2mm} t_\perp = \frac{A'_\perp}{A_\perp} = \frac{2\sin{\varphi'}\cos{\varphi}}{\sin{\left(\varphi'+\varphi\right)}}
\end{equation}
\begin{equation} \label{eq:fresnelr}
  r_\parallel = \frac{A''_\parallel}{A_\parallel} = \frac{\tan{\left(\varphi'-\varphi\right)}}{\tan{\left(\varphi'+\varphi\right)}}, \hspace{2mm} r_\perp = \frac{A''_\perp}{A_\perp} = \frac{\sin{\left(\varphi'-\varphi\right)}}{\sin{\left(\varphi'+\varphi\right)}} \\
\end{equation}

on $\varphi$ és l'angle d'incidència i $\varphi'$ és l'angle de transmissió calculat segons la llei de Snell.
\fi
Mitjançant el mòdul de Python adjunt \texttt{Fresnel.py}, s'ha calculat la variació dels coeficients de Fresnel en el rang angular $\left[\SI{0}{\degree}, \SI{90}{\degree}\right]$ (en els comentaris del propi codi font està explicat el seu funcionament). Les dades obtingudes s'han bolcat en un fitxer \texttt{.csv} i posteriorment s'han representat usant l'eina PGFplots de \LaTeX.

En la figura \ref{fig:Fresnel_coeffs} es representen els valors obtinguts dels coeficients de Fresnel en funció de l'angle d'incidència. Notem que per incidència normal ($\varphi = \SI{0}{\degree}$), tenim que
\begin{equation} \label{eq:incidencianormal}
  t_\parallel = t_\perp = \frac{2n}{n+n'}, \hspace{2mm} r_\parallel = r_\perp = \frac{n-n'}{n+n'}
\end{equation}
i que en l'angle de Brewster $\varphi_\textup{B}$ el coeficient $r_\parallel$ s'anul·la.

\input{Figures/Fresnel_coeffs}

\section*{Aproximació per angles petits}
En la figura \ref{fig:Fresnel_diffs} es representa la diferència entre els coeficients de reflexió paral·lel i perpendicular i l'equivalent per als coeficients de transmissió. Notem que a l'entorn de $\SI{0}{\degree}$, ambdos valors són molt propers a $0$, la qual cosa justifica que en les interferències de làmines plano-paral·leles s'aproximin pel mateix valor.

\begin{figure}[H]
\centering
\begin{tikzpicture}[scale=0.85]
    \begin{axis}[
        width = 14cm,
        height = 7cm,
        enlarge x limits=0.10,
        enlarge y limits=0.10,
        grid = major,
        %minor tick num = 3,
        xlabel={$\varphi$ (\si{\degree})},
        ylabel={},
        xticklabel={$\pgfmathprintnumber{\tick}^\circ$},
        xtick = {0, 15,...,90},
        ytick={0,1,...,2},
        minor xtick={0,5,...,90},
        minor ytick={0,0.75,...,2},
        xmin = 0, xmax = 90, extra x ticks ={},
        ymin = 0, ymax = 2,
        %extra x ticks = {56.768017},
        %extra x tick labels = {{$\varphi_\textup{B}$}},
        %extra x tick style={grid=major, grid style={color=white}, ticklabel pos=top, tick label style={font=\large}},
        %extra y ticks = {-0.20833, 0.79167},
        %extra y tick labels = {$\frac{n-n'}{n+n'}$, $\frac{2n}{n+n'}$},
        %extra y tick style={grid=major, grid style={color=white}, draw=none, font=normalsize},
        %log y ticks with fixed point,
        legend image post style={scale=0.8},
        legend pos=north west,
        table/x index = {0},
        table/col sep = comma,
        /pgf/number format/.cd,
        1000 sep={},
        x tick label style={
        /pgf/number format/.cd,
        fixed,
        fixed zerofill,
        precision=0,
      },
        y tick label style={
        /pgf/number format/.cd,
        fixed,
        fixed zerofill,
        precision=0
      },
    ]
    \addplot[red, thick] plot []table[y index = {5}]{Fresnel.csv};
    \addplot[blue, thick] plot []table[y index = {6}]{Fresnel.csv};
    %\addplot[orange, thick] plot []table[y index = {7}]{Fresnel.csv};
    %\addplot[olive, thick] plot []table[y index = {8}]{Fresnel.csv};

    \addlegendentry{\large $t_\parallel - t_\perp$};
    \addlegendentry{\large $r_\parallel - r_\perp$}

    \draw[black!80!, thick] (axis cs:-10,0) -- (axis cs:100,0);

    %\draw[gray, dotted, thick] (axis cs:56.768017,2) -- (axis cs:56.768017,0);

    %\draw[gray, dotted, thick] (axis cs:-10,-0.20833) -- (axis cs:0,-0.20833);
    %\draw[gray, dotted, thick] (axis cs:-10,0.79167) -- (axis cs:0,0.79167);

    \end{axis}
\end{tikzpicture}
\captionof{figure}{Diferència entre els coeficients de Fresnel per cada component del camp elèctric.}
\label{fig:Fresnel_diffs}
\end{figure}


De fet, podem considerar l'aproximació consistent en prendre el valor de $t_\parallel$ i $t_\perp$ com la mitjana d'aquests dos, ídem per als coeficients de reflexió. En aquest cas, en la figura \ref{fig:Fresnel_approx} veiem que és una aproximació igualment vàlida per angles propers a $\SI{0}{\degree}$. Encara més, mitjançant el programa de generació de les dades, s'ha trobat el rang angular en què l'error relatiu de l'aproximació respecte el valor real és menor al 5\%, obtenint la sortida
\begin{center}
\texttt{> Rang(s) angular(s) en què t\_approx té un error menor al 5\%: [$0.00^\circ$, $58.62^\circ$]\\
> Rang(s) angular(s) en què r\_approx té un error menor al 5\%: [$0.00^\circ$, $15.15^\circ$]}
\end{center}
la qual cosa posa de manifest que els coeficients de Fresnel es poden aproximar satisfactòriament per la seva mitjana en cada component en un rang angular notable, fins al punt que el coeficient de transmissó aproximat presenta un error inferior al 5\% en un rang d'uns \SI{60}{\degree}.
\begin{figure}[H]
\centering
\subfloat[][]{
\begin{tikzpicture}[scale=0.85]
    \begin{axis}[
        width = 10cm,
        height = 8cm,
        enlarge x limits=0.10,
        enlarge y limits=0.10,
        grid = major,
        %minor tick num = 3,
        xlabel={$\varphi$ (\si{\degree})},
        ylabel={},
        xticklabel={$\pgfmathprintnumber{\tick}^\circ$},
        xtick = {0, 15,...,90},
        ytick={-1,0,...,1},
        minor xtick={0,5,...,90},
        minor ytick={-1,-0.75,...,1},
        xmin = 0, xmax = 90, extra x ticks ={},
        ymin = -1, ymax = 1,
        extra x ticks = {56.768017},
        extra x tick labels = {{$\varphi_\textup{B}$}},
        extra x tick style={grid=major, grid style={color=white}, ticklabel pos=top, tick label style={font=\large}},
        extra y ticks = {-0.20833, 0.79167},
        extra y tick labels = {$\frac{n-n'}{n+n'}$, $\frac{2n}{n+n'}$},
        extra y tick style={grid=major, grid style={color=white}, draw=none, font=normalsize},
        %log y ticks with fixed point,
        legend image post style={scale=0.8},
        legend pos=south west,
        table/x index = {0},
        table/col sep = comma,
        /pgf/number format/.cd,
        1000 sep={},
        x tick label style={
        /pgf/number format/.cd,
        fixed,
        fixed zerofill,
        precision=0,
      },
        y tick label style={
        /pgf/number format/.cd,
        fixed,
        fixed zerofill,
        precision=0
      },
    ]
    \addplot[red, thick, dashed, forget plot] plot []table[y index = {1}]{Fresnel.csv};
    \addplot[blue, thick, dashed, forget plot] plot []table[y index = {2}]{Fresnel.csv};
    \addplot[orange, thick, dashed, forget plot] plot []table[y index = {3}]{Fresnel.csv};
    \addplot[olive, thick, dashed, forget plot] plot []table[y index = {4}]{Fresnel.csv};
    \addplot[magenta, very thick] plot []table[y index = {7}]{Fresnel.csv};
    \addplot[brown, very thick] plot []table[y index = {8}]{Fresnel.csv};

    \addlegendentry{\large $\frac{1}{2}\left (t_\parallel + t_\perp\right)$};
    \addlegendentry{\large $\frac{1}{2}\left (r_\parallel + r_\perp\right)$};

    \draw[black!80!, thick] (axis cs:-10,0) -- (axis cs:100,0);

    \draw[gray, dotted, thick] (axis cs:56.768017,2) -- (axis cs:56.768017,0);

    \draw[gray, dotted, thick] (axis cs:-10,-0.20833) -- (axis cs:0,-0.20833);
    \draw[gray, dotted, thick] (axis cs:-10,0.79167) -- (axis cs:0,0.79167);

    \end{axis}
\end{tikzpicture}
}
\subfloat[][]{
\begin{tikzpicture}[scale=0.85]
    \begin{axis}[
        width = 10cm,
        height = 8cm,
        enlarge x limits=0.10,
        enlarge y limits=0.10,
        grid = major,
        %minor tick num = 3,
        xlabel={$\varphi$ (\si{\degree})},
        ylabel={},
        xticklabel={$\pgfmathprintnumber{\tick}^\circ$},
        xtick = {0, 15,...,90},
        ytick={0,1},
        minor xtick={0,5,...,90},
        minor ytick={0,0.1,...,1},
        xmin = 0, xmax = 90, extra x ticks ={},
        ymin = 0, ymax = 1,
        %log y ticks with fixed point,
        legend image post style={scale=0.8},
        legend pos=north west,
        table/x index = {0},
        table/col sep = comma,
        /pgf/number format/.cd,
        1000 sep={},
        x tick label style={
        /pgf/number format/.cd,
        fixed,
        fixed zerofill,
        precision=0,
      },
        y tick label style={
        /pgf/number format/.cd,
        fixed,
        fixed zerofill,
        precision=0
      },
    ]
    \addplot[magenta, very thick] plot []table[y index = {9}]{Fresnel.csv};
    \addplot[brown, very thick] plot []table[y index = {10}]{Fresnel.csv};

    \addlegendentry{\large $\epsilon_t$};
    \addlegendentry{\large $\epsilon_r$};

    \draw[gray, dotted, thick] (axis cs:-10,0.05) -- (axis cs:100,0.05);

    \draw[black!80!, thick] (axis cs:-10,0) -- (axis cs:100,0);

    \end{axis}
\end{tikzpicture}
}
\captionof{figure}{Intensitat detectada pel díode en funció de l'angle de l'escala de l'analitzador, amb la representació de les mitjanes i desviacions estàndards corresponents.}
\label{fig:Circular_I(b)}
\end{figure}


\end{document}
