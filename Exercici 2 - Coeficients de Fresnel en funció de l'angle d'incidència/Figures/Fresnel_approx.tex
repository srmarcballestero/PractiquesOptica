\begin{figure}[H]
\centering
\subfloat[][]{
\begin{tikzpicture}[scale=0.85]
    \begin{axis}[
        width = 10cm,
        height = 8cm,
        enlarge x limits=0.10,
        enlarge y limits=0.10,
        grid = major,
        %minor tick num = 3,
        xlabel={$\varphi$ (\si{\degree})},
        ylabel={},
        xticklabel={$\pgfmathprintnumber{\tick}^\circ$},
        xtick = {0, 15,...,90},
        ytick={-1,0,...,1},
        minor xtick={0,5,...,90},
        minor ytick={-1,-0.75,...,1},
        xmin = 0, xmax = 90, extra x ticks ={},
        ymin = -1, ymax = 1,
        extra x ticks = {56.768017},
        extra x tick labels = {{$\varphi_\textup{B}$}},
        extra x tick style={grid=major, grid style={color=white}, ticklabel pos=top, tick label style={font=\large}},
        extra y ticks = {-0.20833, 0.79167},
        extra y tick labels = {$\frac{n-n'}{n+n'}$, $\frac{2n}{n+n'}$},
        extra y tick style={grid=major, grid style={color=white}, draw=none, font=normalsize},
        %log y ticks with fixed point,
        legend image post style={scale=0.8},
        legend pos=south west,
        table/x index = {0},
        table/col sep = comma,
        /pgf/number format/.cd,
        1000 sep={},
        x tick label style={
        /pgf/number format/.cd,
        fixed,
        fixed zerofill,
        precision=0,
      },
        y tick label style={
        /pgf/number format/.cd,
        fixed,
        fixed zerofill,
        precision=0
      },
    ]
    \addplot[red, thick, dashed, forget plot] plot []table[y index = {1}]{Fresnel.csv};
    \addplot[blue, thick, dashed, forget plot] plot []table[y index = {2}]{Fresnel.csv};
    \addplot[orange, thick, dashed, forget plot] plot []table[y index = {3}]{Fresnel.csv};
    \addplot[olive, thick, dashed, forget plot] plot []table[y index = {4}]{Fresnel.csv};
    \addplot[magenta, very thick] plot []table[y index = {7}]{Fresnel.csv};
    \addplot[brown, very thick] plot []table[y index = {8}]{Fresnel.csv};

    \addlegendentry{\large $\frac{1}{2}\left (t_\parallel + t_\perp\right)$};
    \addlegendentry{\large $\frac{1}{2}\left (r_\parallel + r_\perp\right)$};

    \draw[black!80!, thick] (axis cs:-10,0) -- (axis cs:100,0);

    \draw[gray, dotted, thick] (axis cs:56.768017,2) -- (axis cs:56.768017,0);

    \draw[gray, dotted, thick] (axis cs:-10,-0.20833) -- (axis cs:0,-0.20833);
    \draw[gray, dotted, thick] (axis cs:-10,0.79167) -- (axis cs:0,0.79167);

    \end{axis}
\end{tikzpicture}
}
\subfloat[][]{
\begin{tikzpicture}[scale=0.85]
    \begin{axis}[
        width = 10cm,
        height = 8cm,
        enlarge x limits=0.10,
        enlarge y limits=0.10,
        grid = major,
        %minor tick num = 3,
        xlabel={$\varphi$ (\si{\degree})},
        ylabel={},
        xticklabel={$\pgfmathprintnumber{\tick}^\circ$},
        xtick = {0, 15,...,90},
        ytick={0,1},
        minor xtick={0,5,...,90},
        minor ytick={0,0.1,...,1},
        xmin = 0, xmax = 90, extra x ticks ={},
        ymin = 0, ymax = 1,
        %log y ticks with fixed point,
        legend image post style={scale=0.8},
        legend pos=north west,
        table/x index = {0},
        table/col sep = comma,
        /pgf/number format/.cd,
        1000 sep={},
        x tick label style={
        /pgf/number format/.cd,
        fixed,
        fixed zerofill,
        precision=0,
      },
        y tick label style={
        /pgf/number format/.cd,
        fixed,
        fixed zerofill,
        precision=0
      },
    ]
    \addplot[magenta, very thick] plot []table[y index = {9}]{Fresnel.csv};
    \addplot[brown, very thick] plot []table[y index = {10}]{Fresnel.csv};

    \addlegendentry{\large $\epsilon_t$};
    \addlegendentry{\large $\epsilon_r$};

    \draw[gray, dotted, thick] (axis cs:-10,0.05) -- (axis cs:100,0.05);

    \draw[black!80!, thick] (axis cs:-10,0) -- (axis cs:100,0);

    \end{axis}
\end{tikzpicture}
}
\captionof{figure}{Intensitat detectada pel díode en funció de l'angle de l'escala de l'analitzador, amb la representació de les mitjanes i desviacions estàndards corresponents.}
\label{fig:Circular_I(b)}
\end{figure}
