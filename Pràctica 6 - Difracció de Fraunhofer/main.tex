\documentclass[11pt,a4paper,catalan]{article}
\author{Marc Ballestero Ribó}

%___PAQUETS NECESSARIS___%
\usepackage[utf8]{inputenc}
\usepackage{amsthm,amssymb,amsmath,mathrsfs}
\usepackage[catalan]{babel}
\usepackage{wasysym}
\usepackage[margin=25mm]{geometry}
\usepackage[bottom]{footmisc}
\usepackage{enumerate}
\usepackage{array, boldline, makecell, booktabs, dcolumn}
\usepackage[svgnames, table]{xcolor}
\usepackage[all]{xy}
\usepackage{parskip}
\usepackage{float, subfig}
\usepackage{fancyhdr}
\usepackage{graphicx}
\usepackage{multirow,multicol}
\usepackage{pgfplots, tikz}
\usepackage{pgfplotstable}
\usepackage{graphics}
\usepackage{arydshln}
\usepackage{caption}
\usepackage{siunitx}
\usepackage{chngpage}
\usepackage{enumitem}
\usepackage{hyperref}
\hypersetup{
    colorlinks=true,
    linkcolor=black,
    filecolor=black,
    urlcolor=blue,
}

\sisetup{separate-uncertainty=true, exponent-product=\cdot}
\DeclareSIUnit\dioptre{D}
\DeclareSIUnit\arbitrary{u.a.}
\DeclareSIUnit\pixel{px}
\DeclareSIUnit\bit{bit}
\DeclareSIUnit\percent{\%}
\decimalpoint



%\sisetup{locale = FR}

%___NOUS TIPUS DE COLUMNES PER LES TAULES___%
\newcolumntype{L}[1]{>{\raggedright\let\newline\\\arraybackslash\hspace{0pt}}m{#1}}
\newcolumntype{C}[1]{>{\centering\let\newline\\\arraybackslash\hspace{0pt}}m{#1}}
\newcolumntype{R}[1]{>{\raggedleft\let\newline\\\arraybackslash\hspace{0pt}}m{#1}}
\newcolumntype{d}[1]{D{.}{.}{#1}}

%___COMANDAMENTS PERSONALITZATS___%
\providecommand{\keywords}[1]{\textsc{{Paraules clau---}} #1}
\newcommand{\diff}{\mathrm{d}}
\DeclareMathOperator{\sinc}{sinc}
\DeclareMathOperator{\rect}{rect}
\DeclareMathOperator{\fcirc}{circ}
\newcommand{\nind}{\noindent}
\newcommand{\res}[3]{$\left ( #1 \pm #2 \right ) \, \si{#3}$}
\newcommand{\figref}[1]{figura \ref{#1}}
\newcommand{\tabref}[1]{taula \ref{#1}}
\renewcommand\thesection{\arabic{section}}
\newcommand{\data}{\today}


%___CAPÇALERA I PEU DE PÀGINA___%
\pagestyle{fancy}
\headheight=16pt
\setlength{\headsep}{1cm}
\fancypagestyle{headings}{
\cfoot{\thepage}}
\lhead{\scshape Òptica \textup{2020-21}}
\rhead{\data \\ Marc Ballestero Ribó}

%___CONFIGURACIÓ DEL PGFPLOTS___%
\pgfplotsset{compat = 1.15}
\pgfplotsset{
    legend image with text/.style={
        legend image code/.code={%
            \node[anchor=center] at (0.3cm,0cm) {#1};
        }
    },
    log x ticks with fixed point/.style={
      xticklabel={
        \pgfkeys{/pgf/fpu=true}
        \pgfmathparse{exp(\tick)}%
        \pgfmathprintnumber[fixed relative, precision=0]{\pgfmathresult}
        \pgfkeys{/pgf/fpu=false}
      }
    },
    log y ticks with fixed point/.style={
      yticklabel={
        \pgfkeys{/pgf/fpu=true}
        \pgfmathparse{exp(\tick)}%
        \pgfmathprintnumber[fixed zerofill, precision=0]{\pgfmathresult}
        \pgfkeys{/pgf/fpu=false}
      }
  },
}
\pgfplotsset{every tick label/.append style={font=\footnotesize}}
\usetikzlibrary{spy}

%___CONFIGURACIÓ DELS PEUS DE FIGURA___%
\captionsetup[figure]{justification = justified,labelfont={small,sc},textfont=small}
\captionsetup[table]{name = Taula, justification = justified,labelfont={small,sc},textfont=small}

%___DIRECTORI D'IMATGES___%
\graphicspath{ {./Imatges/} }




%___TÍTOL___%

\title{\vspace{-1.5cm} Òptica \\ \vspace{1mm} Pràctica 6. Difracció de Fraunhofer}
\author{Marc Ballestero Ribó - Grup D2}
\date{\data}


\begin{document}
\maketitle

\begin{abstract}
 \nind L'objectiu d'aquesta pràctica és l'obtenció i anàlisi experimental de figures de difracció de Fraunhofer, tot comprovant la validesa del model teòric. Mitjançant un làser díode, un conjunt de lents i un sensor CCD, s'ha estudiat el comportament difractiu de la llum quan es propaga a través d'una obertura rectangular i una de circular, així com quan travessa obertures rectangulars i circulars dobles. A partir de l'ajust dels perfils d'intensitat lumínica obtinguts al model teòric de Fraunhofer, s'han pogut estimar experimentalment els paràmetres geomètrics característics de les obertures, amb un grau de precisió notable.
\end{abstract}

\section{Dispositiu experimental} \label{sec:disp}
El sistema experimental utilitzat consisteix en un banc òptic sobre el qual s'hi ha col·locat un làser díode de longitud d'ona $\lambda = \SI{635}{\nano\metre}$ i potència d'emissió regulable, seguit d'una lent difusora $\textup{L}_1$ situada de tal manera que el seu pla focal objecte coincideixi amb el punt d'emissió del làser, és a dir, les ones que s'hi propaguen esdevenen planes. A continuació, sobre un peu s'hi han col·locat les diferents obertures de difracció $\Sigma$. A partir d'aquest punt, la llum es comporta segons la difracció de Fresnel. Per tal d'aconseguir difracció de Fraunhofer, s'ha col·locat una lent de Fourier $\textup{L}_2$ de focal $f' = \SI{350}{\milli\metre}$ després de l'objecte. Finalment, un sensor CCD de resolució $640 \times 480 \, \si{\pixel}$ i mida de píxel $\SI{6}{\micro\metre}$ s'ha col·locat en el pla focal imatge de $\textup{L}_2$ per a captar les ones difractades. La profunditat de color del sensor és de $\SI{8}{\bit}$, i, atès que la seva resposta és lineal, s'ha considerat que la intensitat ve donada en unitats arbitràries (\si{\arbitrary}) proporcionades pel sensor. La potència del làser s'ha ajustat convenientment per tal de no saturar el sensor.

Finalment, un ordinador connectat a la CCD permet bolcar les dades i realitzar els ajustos corresponents.

A la figura \ref{fig:disp} es detalla un esquema del dispositiu experimental.

\begin{figure}[H]
  \centering
  \includegraphics[height = 5cm]{Imatges/Disp.png}
\captionof{figure}{Dispositiu experimental.}
\label{fig:disp}
\end{figure}




\section{Perfils d'intensitat i figures de difracció} \label{sec:figdif}
En aquesta secció s'estudien les diferents figures de difracció obtingudes per a cada tipus d'obertura, tot calculant els seus paràmetres geomètrics a partir dels perfils d'intensitats obtinguts i els ajustos corresponents.

\subsection{Obertura rectangular} \label{sec:rect1}
Un objecte de difracció consistent en una petita obertura rectangular de dimensions $L_x \times L_y$ es pot modelar mitjançant la \emph{funció rectangle bidimensional}, $f\left(x,y\right) = \rect\left(x/L_x, y/L_y\right)$, que té transmitància 1 per $\left(x,y\right) \in \left(-L_x/2, L_x/2\right) \times \left(-L_y/2, L_y/2\right)$ i 0 altrament. Així, la distribució d'intensitats de la difracció de Fraunhofer de l'objecte queda\footnote{Definim la funció \emph{sinus quocient} $\sinc : \mathbb{R} \to \mathbb{R}$ com $\sinc{x} := \sin{\left(\pi x\right)}/\left(\pi x\right),$ per $x \in \mathbb{R} \smallsetminus \{0\} $ i $ \sinc{0} := 1,$ per $x = 0$.}
\begin{equation} \label{eq:Irect}
 I\left(x,y\right) \propto \left [\sinc{\left(\frac{L_x x}{\lambda f'}\right)}\,\sinc{\left(\frac{L_y y}{\lambda f'}\right)}\right]^2
\end{equation}
on $\lambda$ és la longitud d'ona de la font que es propaga per l'escletxa i $f'$ la focal de la lent de Fourier $\textup{L}_2$.

\begin{figure}[H]
  \centering
  \includegraphics[height = 5cm]{Imatges/Rect1.png}
\captionof{figure}{Distribució d'intensitats capturada pel sensor CCD en el pla d'observació, per a l'obertura rectangular.}
\label{fig:Rect1-Img}
\end{figure}


Els perfils d'intensitats en cada eix $I(x) := I(x,0)$ i $I(y) := I(0,y)$ obtinguts mitjançant l'anàlisi de les dades proporcionades pel sensor CCD es representen en la figura \ref{fig:Rect1-XY}, així com les funcions ajustades per superposició amb la gràfica experimental, donades per
\begin{equation} \label{eq:irect}
 \begin{aligned}
  I(x) & = h_x \left [\sinc\left(\Lambda_x \left (x - \Delta x\right)\right) \cos\left(T_x x\right) \right]^2 \\
  I(y) & = h_y \left [\sinc\left(\Lambda_y \left (y - \Delta y\right)\right) \cos\left(T_y y\right) \right]^2
 \end{aligned}
\end{equation}
que depenen dels paràmetres ajustats manualment sobre la corba empírca $h_x, h_y$, anomenats \emph{alçada}, $\Lambda_x, \Lambda_y$, als quals es farà referència per \emph{amplada} i $\Delta x, \Delta y$, que es designen per \emph{desplaçament}.

Tenint en compte que, per al sensor CCD usat, $\SI{1}{\pixel} = \SI{6}{\micro\metre}$, per a l'obertura rectangular s'han obtingut els paràmetres que es detallen en la taula \ref{tab:Rect1-Dat}. S'ha considerat que la seva incertesa està en l'última xifra decimal donada pel \emph{software} de recollida de dades.

\begin{table}[H]
\centering
\renewcommand{\arraystretch}{1.2}
    \begin{tabular}{c r | r}
        \hline
        $h_x$ & (\si{\milli\metre}) & \num{1.530(6)} \\
        $\Lambda_x$ & (\si{\per\milli\metre})& \num{2.317(17)} \\
        $\Delta x$ & (\si{\milli\metre}) & \num{-0.024(6)} \\
        $T_x$ & (\si{\per\milli\metre}) & \num{0.000(17)} \\
        \hline
    \end{tabular}
    \hspace{5mm}
    \begin{tabular}{c r | r}
        \hline
        $h_y$ & (\si{\milli\metre}) & \num{1.530(6)} \\
        $\Lambda_y$ & (\si{\per\milli\metre})& \num{8.117(17)} \\
        $\Delta y$ & (\si{\milli\metre}) & \num{0.000(6)} \\
        $T_y$ & (\si{\per\milli\metre}) & \num{0.000(17)} \\
        \hline
    \end{tabular}
\captionof{table}{Paràmetres de l'ajust per a l'obertura rectangular.}
\label{tab:Rect1-Dat}
\renewcommand{\arraystretch}{1}
\end{table}


Així, comparant \eqref{eq:Irect} amb \eqref{eq:irect}, es poden obtenir les dimensions de l'escletxa a partir dels paràmetres de l'ajust segons
\begin{equation} \label{eq:LsRect}
 \begin{aligned}
  L_x & = \Lambda_x \lambda f' \\
  L_y & = \Lambda_y \lambda f'
 \end{aligned}
\end{equation}
, amb la qual cosa s'obté que $L_x = \SI{0.527(4)}{\milli\metre}$ i $L_y = \SI{1.846(4)}{\milli\metre}$.
\begin{figure}[H]
  \centering
  \subfloat[][]{
    \input{Figures/Rect1 - X}
  }
  \hspace{5mm}
  \subfloat[][]{
    \begin{tikzpicture}[]
\begin{axis}[
    width = 12cm,
    height = 7cm,
    enlarge x limits=0.10,
    enlarge y limits=0.10,
    %minor tick num = 3,
    xlabel={$y$ (\si{\milli\metre})},
    ylabel={$I$ (\si{\arbitrary})},
    xtick={-1.5,-1.0,...,1.5},
    %scaled x tick = {real:3.141516},
    ytick={0,50,...,300},
    %xticklabels={$-3$, $-2$, $-1$, $0$, $1$, $2$, $3$},
    minor xtick={-2,-1.9,...,2},
    minor ytick={-30,-20,...,330},
    xmin = -1.5, xmax = 1.5, extra x ticks ={},
    ymin = 0, ymax = 300,
    %extra y ticks ={4.18},
    %ymode = log,
    %log y ticks with fixed point,
    legend image post style={scale=0.5},
    legend pos=north west,
    /pgf/number format/.cd,
    1000 sep={},
    x tick label style={
    /pgf/number format/.cd,
    fixed,
    fixed zerofill,
    precision=1
  },
    y tick label style={
    /pgf/number format/.cd,
    fixed,
    fixed zerofill,
    precision=0
  },
]
\addplot[red, thick, scatter, scatter/classes={a={mark=diamond, red!80!, scale=0.5},b={mark=x,black,scale=1.2}}, scatter src=explicit symbolic
    ] plot []table[x = x, y = I, meta=class]{Dades/Rect1 - Col.csv};
\addplot[blue, thick, opacity=0.6, scatter, scatter/classes={a={mark=+, blue!80!, scale=0.5, opacity=0.6},b={mark=x,black,scale=1.2}}, scatter src=explicit symbolic
        ] plot []table[x = x, y = i, meta=class]{Dades/Rect1 - Col.csv};
\end{axis}
\end{tikzpicture}
\label{fig:Rect1-Y}

  }
\captionof{figure}{(\textsc{a}) Intensitats detectada i ajustada en funció de la posició sobre l'eix horitzontal central $X$ i (\textsc{b}) sobre l'eix vertical central $Y$ del sensor CCD, per a l'obertura rectangular.}
\label{fig:Rect1-XY}
\end{figure}


\subsection{Obertura circular} \label{sec:circ1}
En aquest cas, el feix de llum làser es propaga a través d'una obertura circular de radi $R$, que en coordenades polars es pot modelitzar per $f(r) = \fcirc{\left(r/R\right)}$, funció de transmitància 1 per $r \in \left(0, R\right)$ i 0 altrament. Així doncs, el seu perfil de difracció ve donat per
\begin{equation} \label{eq:Icirc1}
 I(r) \propto \left[\frac{2 J_1 \left(\pi \beta\right)}{\pi \beta}\right]^2, \hspace{3mm} \beta := \frac{2Rr}{\lambda f'}
\end{equation}
on $J_1$ és la funció de Bessel de primera espècie i ordre 1.

\begin{figure}[H]
  \centering
  \includegraphics[height = 5cm]{Imatges/Circ1.png}
\captionof{figure}{Distribució d'intensitats capturada pel sensor CCD en el pla d'observació, per a l'obertura circular.}
\label{fig:Circ1-Img}
\end{figure}


Mitjançant l'ajust adient del perfil d'intensitats, realitzat amb el \emph{software} de recollida de dades i representat en la figura \ref{fig:Circ1-XY}, s'obtenen els paràmetres de la taula \ref{tab:Circ1-Dat}. En aquest cas, observem que $\Lambda_x$ i $\Lambda_y$ són compatibles dins el seu marge d'intcertesa, fet coherent amb la simetria radial de l'obertura circular.

\begin{table}[H]
\centering
\renewcommand{\arraystretch}{1.2}
    \begin{tabular}{c r | r}
        \hline
        $h_x$ & (\si{\arbitrary}) & \num{255(1)} \\
        $\Lambda_x$ & (\si{\per\milli\metre})& \num{2.450(17)} \\
        $\Delta x$ & (\si{\milli\metre}) & \num{0.000(6)} \\
        $T_x$ & (\si{\per\milli\metre}) & \num{0.000(17)} \\
        \hline
    \end{tabular}
    \hspace{5mm}
    \begin{tabular}{c r | r}
        \hline
        $h_y$ & (\si{\arbitrary}) & \num{255(1)} \\
        $\Lambda_y$ & (\si{\per\milli\metre})& \num{2.433(17)} \\
        $\Delta y$ & (\si{\milli\metre}) & \num{-0.030(6)} \\
        $T_y$ & (\si{\per\milli\metre}) & \num{0.000(17)} \\
        \hline
    \end{tabular}
\captionof{table}{Paràmetres de l'ajust per a l'obertura circular.}
\label{tab:Circ1-Dat}
\renewcommand{\arraystretch}{1}
\end{table}


Finalment, el radi de l'obertura es pot obtenir a partir de cadascuna de les amplades de manera independent, segons
\begin{align}
 \label{eq:Rx}
 R & = \frac{\Lambda_x \lambda f'}{2} \\
 \label{eq:Ry}
 R & = \frac{\Lambda_y \lambda f'}{2}
\end{align}
.

Si denotem $R_x$ i $R_y$ els valors calculats per \eqref{eq:Rx} i \eqref{eq:Ry}, respectivament; obtenim que $R_x = \SI{0.2787(19)}{\milli\metre}$ i $R_y = \SI{0.2767(19)}{\milli\metre}$, que són clarament compatibles. S'estima, doncs, el radi de l'escletxa com la mitjana aritmètica  d'aquests valors junt amb la corresponent incertesa propagada, d'on $R = \SI{0.2777(13)}{\milli\metre}$.

\begin{figure}[H]
  \centering
  \subfloat[][]{
    \input{Figures/Circ1 - X}
  }
  \hspace{5mm}
  \subfloat[][]{
    \input{Figures/Circ1 - Y}
  }
\captionof{figure}{(\textsc{a}) Intensitats detectada i ajustada en funció de la posició sobre l'eix horitzontal central $X$ i (\textsc{b}) sobre l'eix vertical central $Y$ del sensor CCD, per a l'obertura circular.}
\label{fig:Circ1-XY}
\end{figure}


\subsection{Doble obertura rectangular} \label{sec:Rect2}
Si el feix de llum es propaga a través de dues obertures rectangulars paral·leles amb els seus centres separats una distància $D$ en la direcció de l'eix $X$, la distribució d'intensitats de la difracció de Fraunhofer en el pla d'observació ve donada per
\begin{equation}
   I\left(x,y\right) \propto \left [\sinc{\left(\frac{L_x x}{\lambda f'}\right)}\,\sinc{\left(\frac{L_y y}{\lambda f'}\right)} \, \cos{\left(\frac{\pi D}{\lambda f'} x\right)}\right]^2
\end{equation}
és a dir, el perfil \eqref{eq:Irect} modulat pel cosinus al quadrat d'una quantitat proporcional a la separació entre els rectangles i la posició en $X$. Notem que aquest terme s'atribueix a l'aparició de franjes en el pla d'observació, fàcilment relacionables amb les interferències de Young.

\begin{figure}[H]
  \centering
  \includegraphics[height = 5cm]{Imatges/Rect2.png}
\captionof{figure}{Distribució d'intensitats capturada pel sensor CCD en el pla d'observació, per a la doble obertura rectangular. S'observa l'aparició de franjes d'interferències de Young.}
\label{fig:Rect2-Img}
\end{figure}


Ajustant el perfil d'intensitats, representat a la figura \ref{fig:Rect2-XY}, amb el \emph{software} d'anàlisi de dades, s'obtenen els paràmetres de la taula \ref{tab:Rect2-Dat}.

\begin{table}[H]
\centering
\renewcommand{\arraystretch}{1.2}
    \begin{tabular}{c r | r}
        \hline
        $h_x$ & (\si{\arbitrary}) & \num{255(1)} \\
        $\Lambda_x$ & (\si{\per\milli\metre})& \num{1.967(17)} \\
        $\Delta x$ & (\si{\milli\metre}) & \num{0.000(6)} \\
        $T_x$ & (\si{\per\milli\metre}) & \num{5.017(17)} \\
        \hline
    \end{tabular}
    \hspace{5mm}
    \begin{tabular}{c r | r}
        \hline
        $h_y$ & (\si{\arbitrary}) & \num{255(1)} \\
        $\Lambda_y$ & (\si{\per\milli\metre})& \num{11.117(17)} \\
        $\Delta y$ & (\si{\milli\metre}) & \num{0.000(6)} \\
        $T_y$ & (\si{\per\milli\metre}) & \num{0.000(17)} \\
        \hline
    \end{tabular}
\captionof{table}{Paràmetres de l'ajust per a la doble obertura rectangular.}
\label{tab:Rect2-Dat}
\renewcommand{\arraystretch}{1}
\end{table}


Notem que el període $T_x$ és no nul. Així doncs, tenint en compte \eqref{eq:irect}, les dimensions de les escletxes venen donades per \eqref{eq:LsRect}, mentres que la seva separació es calcula segons
\begin{equation} \label{eq:D}
  D = \frac{1}{\pi} T_x \lambda f'
\end{equation}
.

En conseqüència, s'obté que $L_x = \SI{0.447(4)}{\milli\metre}$, $L_y = \SI{2.529(4)}{\milli\metre}$ i la distància de separació és $D = \SI{0.3549(12)}{\milli\metre}$.

\begin{figure}[H]
  \centering
  \subfloat[][]{
    \input{Figures/Rect2 - X}
  }
  \hspace{5mm}
  \subfloat[][]{
    \input{Figures/Rect2 - Y}
  }
\captionof{figure}{(\textsc{a}) Intensitats detectada i ajustada en funció de la posició sobre l'eix horitzontal central $X$ i (\textsc{b}) sobre l'eix vertical central $Y$ del sensor CCD, per a la doble obertura rectangular.}
\label{fig:Rect2-XY}
\end{figure}


\subsection{Doble obertura circular} \label{sec:circ2}
Finalment, s'analitza el compartament de la llum quan es propaga a través de dues obertures circulars de radi $R$ separades una distància $D$ en l'eix $X$. De manera anàloga al cas anterior, es té que
\begin{equation} \label{eq:Icirc2}
   I(r) \propto \left[\frac{2 J_1 \left(\pi \beta\right)}{\pi \beta}\, \cos{\left(\frac{\pi D}{\lambda f'} x\right)}\right]^2, \hspace{3mm} \beta := \frac{2Rr}{\lambda f'}
\end{equation}
és a dir, el producte del perfil d'intensitats \eqref{eq:Icirc1} d'un sol cercle modulat pel terme cosinus quadrat relacionat amb les interferències de Young.

\begin{figure}[H]
  \centering
  \includegraphics[height = 5cm]{Imatges/Circ2.png}
\captionof{figure}{Distribució d'intensitats capturada pel sensor CCD en el pla d'observació, per a la doble obertura circular.}
\label{fig:Circ2-Img}
\end{figure}


Si s'ajusta el perfil d'intensitats, que es representa gràficament a la figura \ref{fig:Circ2-XY}, s'obtenen els paràmetres de la taula \ref{tab:Circ2-Dat}.

\begin{table}[H]
\centering
\renewcommand{\arraystretch}{1.2}
    \begin{tabular}{c r | r}
        \hline
        $h_x$ & (\si{\arbitrary}) & \num{255(1)} \\
        $\Lambda_x$ & (\si{\per\milli\metre})& \num{3.367(17)} \\
        $\Delta x$ & (\si{\milli\metre}) & \num{0.000(6)} \\
        $T_x$ & (\si{\per\milli\metre}) & \num{10.150(17)} \\
        \hline
    \end{tabular}
    \hspace{5mm}
    \begin{tabular}{c r | r}
        \hline
        $h_y$ & (\si{\arbitrary}) & \num{255(1)} \\
        $\Lambda_y$ & (\si{\per\milli\metre})& \num{3.367(17)} \\
        $\Delta y$ & (\si{\milli\metre}) & \num{0.000(6)} \\
        $T_y$ & (\si{\per\milli\metre}) & \num{0.000(17)} \\
        \hline
    \end{tabular}
\captionof{table}{Paràmetres de l'ajust per a la doble obertura circular.}
\label{tab:Circ2-Dat}
\renewcommand{\arraystretch}{1}
\end{table}


Notem que, d'una banda, el període $T_x$ no és nul, i de l'altra, les amplades en $x$ i $y$ coincideixen, tal com s'espera donada la simetria radial del cercle.

A partir de \eqref{eq:Icirc2}, tenim que el radi dels cercles vindrà donat per \eqref{eq:Rx} i \eqref{eq:Ry}, d'on s'obté que $R_x = \SI{0.3830(19)}{\milli\metre}$ i $R_y = \SI{0.3830(19)}{\milli\metre}$, amb la qual cosa s'estima que el radi de les obetures és la mitjana $R = \SI{0.3830(13)}{\milli\metre}$. La distància de separació entre les obertures s'obté aplicant \eqref{eq:D}, d'on $D = \SI{0.7181(12)}{\milli\metre}$.

\begin{figure}[H]
  \centering
  \subfloat[][]{
    \begin{tikzpicture}[scale=0.90]
\begin{axis}[
    enlarge x limits=0.10,
    enlarge y limits=0.10,
    %minor tick num = 3,
    xlabel={$x$ (\si{\milli\metre})},
    ylabel={$I$ (\si{\arbitrary})},
    xtick={-2,-1.5,...,2},
    %scaled x tick = {real:3.141516},
    ytick={0,50,...,300},
    %xticklabels={$-3$, $-2$, $-1$, $0$, $1$, $2$, $3$},
    minor xtick={-2.5,-2.4,...,2.5},
    minor ytick={-30,-20,...,330},
    xmin = -2, xmax = 2, extra x ticks ={},
    ymin = 0, ymax = 300,
    %extra y ticks ={4.18},
    %ymode = log,
    %log y ticks with fixed point,
    legend image post style={scale=0.5},
    legend pos=north east,
    /pgf/number format/.cd,
    1000 sep={},
    x tick label style={
    /pgf/number format/.cd,
    fixed,
    fixed zerofill,
    precision=1
  },
    y tick label style={
    /pgf/number format/.cd,
    fixed,
    fixed zerofill,
    precision=0
  },
]
\addplot[red, thick, scatter, scatter/classes={a={mark=diamond, red!80!, scale=0.5}}, scatter src=explicit symbolic] plot []table[x = x, y = I, meta=class]{Dades/Circ2 - Fil.csv};
\addplot[blue, thick, opacity=0.6, scatter, scatter/classes={a={mark=+, blue!80!, scale=0.5, opacity=0.6},b={mark=x,black,scale=1.2}}, scatter src=explicit symbolic] plot []table[x = x, y = i, meta=class]{Dades/Circ2 - Fil.csv};
\addlegendentry{\footnotesize Experimental}
\addlegendentry{\footnotesize Ajust}
\end{axis}
\end{tikzpicture}
\label{fig:Circ2-X}

  }
  \hspace{5mm}
  \subfloat[][]{
    \input{Figures/Circ2 - Y}
  }
\captionof{figure}{(\textsc{a}) Intensitats detectada i ajustada en funció de la posició sobre l'eix horitzontal central $X$ i (\textsc{b}) sobre l'eix vertical central $Y$ del sensor CCD, per a la doble obertura circular.}
\label{fig:Circ2-XY}
\end{figure}


\appendix
\newpage
\section{Fórmules estadístiques}\label{sec:apendix}
\setcounter{equation}{0}
\renewcommand{\theequation}{\thesection.\arabic{equation}}
%\renewcommand{\thesubsection}{\thesection.\Roman{subsection}}
\subsection{Paràmetres centrals i de dispersió}
Donada una mostra de $N$ elements $\left\{x_1,\dots,x_N\right\}$, definim els següents paràmetres estadístics.
\begin{itemize}[label=--]
    \item Mitjana aritmètica \begin{equation}\label{est:mitjana}
            \langle x\rangle =\frac{1}{N} \sum_{i=1}^{N} x_i
        \end{equation}
    \item Desviació estàndard\footnote{S'ha fet servir la correcció de Bessel $\sqrt{N/(N-1)}$ de la desviació estàndard poblacional.} \begin{equation}\label{est:stdev}
            {\displaystyle \sigma_x ={\sqrt {{\frac {1}{N-1}}\sum _{i=1}^{N}(x_{i}-\langle x \rangle )^{2}}}}
        \end{equation}
    \item Error estàndard \begin{equation}\label{est:err}
   {\displaystyle \delta{x} ={\frac {\sigma_x}{\sqrt {N}}}}
\end{equation}

\end{itemize}
\subsection{Estimacions lineals}
Per al càlcul de les estimacions lineals s'usa la funció \texttt{ESTIMACION.LINEAL} del full de càlcul \texttt{Microsoft Excel}, que proporciona el pendent i l'ordenada a l'origen de la recta de regressió amb les seves corresponents incerteses, així com el coeficient de correlació $R^2$ i l'error estàndard de la regressió.
% El funcionament complet de la funció es pot consultar a \url{https://support.office.com/es-es/article/estimacion-lineal-funci\%C3\%B3n-estimacion-lineal-84d7d0d9-6e50-4101-977a-fa7abf772b6d}.

\subsection{Test \texorpdfstring{$\chi^2$}{x2}}
Donat un ajust lineal $y=A\,x+B$, amb incertesa en la variable dependent $\delta y$ i error estàndard de la regressió $\delta y_\textup{reg}$, es defineix el \emph{coeficient $\chi^2$} com
\begin{equation}
    \chi^2 = \nu\, \left(\frac{\delta y_\textup{reg}}{\delta y}\right)
\end{equation}
on $\nu$ es el nombre de graus de llibertat de l'ajust. Amb això, podem definir el \emph{coeficient reduït $\chi^2_\nu$} com
\begin{equation}
    \chi^2_\nu = \frac{\delta y_\textup{reg}}{\delta y}
\end{equation}
El valor d'aquest paràmetre ens indica la bondat de l'ajust realitzat. Tenim que
\begin{enumerate}[label=(\alph*)]
    \item si \underline{$\delta y_\textup{reg} \ll \delta y$} o \underline{$\chi^2_\nu \ll 1$}, l'ajust és acceptable i probablement s'hagi sobreestimat $\delta y$;
    \item si \underline{$\delta y_\textup{reg} \lesssim \delta y$} o \underline{$\chi^2_\nu \lesssim 1$}, l'ajust és acceptable; i
    \item si \underline{$\delta y_\textup{reg} \gg \delta y$} o \underline{$\chi^2_\nu \gg 1$} l'ajust no és acceptable.
\end{enumerate}

\end{document}
