\documentclass[11pt,a4paper,catalan]{article}
\author{Marc Ballestero Ribó}

%___PAQUETS NECESSARIS___%
\usepackage[utf8]{inputenc}
\usepackage{amsthm,amssymb,amsmath,mathrsfs}
\usepackage[catalan]{babel}
\usepackage{wasysym}
\usepackage[margin=25mm]{geometry}
\usepackage[bottom]{footmisc}
\usepackage{enumerate}
\usepackage{array, boldline, makecell, booktabs, dcolumn}
\usepackage[svgnames, table]{xcolor}
\usepackage[all]{xy}
\usepackage{parskip}
\usepackage{float, subfig}
\usepackage{fancyhdr}
\usepackage{graphicx}
\usepackage{multirow,multicol}
\usepackage{pgfplots, tikz}
\usepackage{pgfplotstable}
\usepackage{graphics}
\usepackage{arydshln}
\usepackage{caption}
\usepackage{siunitx}
\usepackage{chngpage}
\usepackage{enumitem}
\usepackage{hyperref}
\hypersetup{
    colorlinks=true,
    linkcolor=black,
    filecolor=black,
    urlcolor=blue,
}

\sisetup{separate-uncertainty=true, exponent-product=\cdot}
\DeclareSIUnit\dioptre{D}
\DeclareSIUnit\arbitrary{u.a.}
\DeclareSIUnit\pixel{px}
\DeclareSIUnit\percent{\%}
\decimalpoint



%\sisetup{locale = FR}

%___NOUS TIPUS DE COLUMNES PER LES TAULES___%
\newcolumntype{L}[1]{>{\raggedright\let\newline\\\arraybackslash\hspace{0pt}}m{#1}}
\newcolumntype{C}[1]{>{\centering\let\newline\\\arraybackslash\hspace{0pt}}m{#1}}
\newcolumntype{R}[1]{>{\raggedleft\let\newline\\\arraybackslash\hspace{0pt}}m{#1}}
\newcolumntype{d}[1]{D{.}{.}{#1}}

%___COMANDAMENTS PERSONALITZATS___%
\providecommand{\keywords}[1]{\textsc{{Paraules clau---}} #1}
\newcommand{\diff}{\mathrm{d}}
\DeclareMathOperator{\sinc}{sinc}
\newcommand{\nind}{\noindent}
\newcommand{\res}[3]{$\left ( #1 \pm #2 \right ) \, \si{#3}$}
\newcommand{\figref}[1]{figura \ref{#1}}
\newcommand{\tabref}[1]{taula \ref{#1}}
\renewcommand\thesection{\arabic{section}}
\newcommand{\data}{\today}


%___CAPÇALERA I PEU DE PÀGINA___%
\pagestyle{fancy}
\headheight=16pt
\setlength{\headsep}{1cm}
\fancypagestyle{headings}{
\cfoot{\thepage}}
\lhead{\scshape Òptica \textup{2020-21}}
\rhead{\data \\ Marc Ballestero Ribó}

%___CONFIGURACIÓ DEL PGFPLOTS___%
\pgfplotsset{compat = 1.15}
\pgfplotsset{
    legend image with text/.style={
        legend image code/.code={%
            \node[anchor=center] at (0.3cm,0cm) {#1};
        }
    },
    log x ticks with fixed point/.style={
      xticklabel={
        \pgfkeys{/pgf/fpu=true}
        \pgfmathparse{exp(\tick)}%
        \pgfmathprintnumber[fixed relative, precision=0]{\pgfmathresult}
        \pgfkeys{/pgf/fpu=false}
      }
    },
    log y ticks with fixed point/.style={
      yticklabel={
        \pgfkeys{/pgf/fpu=true}
        \pgfmathparse{exp(\tick)}%
        \pgfmathprintnumber[fixed zerofill, precision=0]{\pgfmathresult}
        \pgfkeys{/pgf/fpu=false}
      }
  },
}
\usetikzlibrary{spy}

%___CONFIGURACIÓ DELS PEUS DE FIGURA___%
\captionsetup[figure]{justification = justified,labelfont={small,sc},textfont=small}
\captionsetup[table]{name = Taula, justification = justified,labelfont={small,sc},textfont=small}

%___DIRECTORI D'IMATGES___%
\graphicspath{ {./Imatges/} }




%___TÍTOL___%

\title{\vspace{-1.5cm} Òptica \\ \vspace{1mm} Pràctica 6. Difracció de Fraunhofer}
\author{Marc Ballestero Ribó - Grup D2}
\date{\data}


\begin{document}
\maketitle

\begin{abstract}
 \nind
\end{abstract}

\section{Perfils d'intensitat i figures de difracció} \label{sec:figdif}
En aquesta secció s'estudien les diferents figures de difracció obtingudes per a cada tipus d'obertura, tot calculant els seus paràmetres geomètrics a partir dels perfils d'intensitats obtinguts i els ajustos corresponents.

\subsection{Obertura rectangular} \label{sec:rect1}
Un objecte de difracció consistent en una petita obertura rectangular de dimensions $L_x \times L_y$ es pot modelar mitjançant la \emph{funció rectangle bidimensional}, $f\left(x,y\right) = \mathrm{rect}\left(x/L_x, y/L_y\right)$, que té transmitància 1 per $\left(x,y\right) \in \left[-L_x/2, L_x/2\right] \times \left[-L_y/2, L_y/2\right]$ i 0 altrament. Així, la distribució d'intensitats de la difracció de Fraunhofer de l'objecte queda\footnote{Definim la funció \emph{sinus quocient} $\sinc : \mathbb{R} \to \mathbb{R}$ com $\sinc{x} := \sin{\left(\pi x\right)}/\left(\pi x\right),$ per $x \in \mathbb{R} \smallsetminus \{0\} $ i $ \sinc{0} := 1,$ per $x = 0$.}
 \begin{equation}
  I\left(x,y\right) \propto \left [\sinc{\left(\frac{L_x x}{\lambda f'}\right)}\,\sinc{\left(\frac{L_y y}{\lambda f'}\right)}\right]^2
 \end{equation}
 on $\lambda$ és la longitud d'ona de la font que es propaga per l'escletxa i $f'$ la focal de la lent de Fourier $\textup{L}_2$.

 Els perfils d'intensitats en cada eix $I(x) := I(x,0)$ i $I(y) := I(0,y)$ obtinguts mitjançant l'anàlisi de les dades volcades pel sensor CCD es representen en la figura \ref{fig:Rect1-XY}, així com les funcions ajustades per superposició amb la gràfica experimental, donades per
 \begin{align}
   I(x) &= h_x \left [\sinc\left(\Lambda_x \left (x - \Delta x\right)\right) \cos\left(T_x x\right) \right]^2 \\
   I(y) &= h_y \left [\sinc\left(\Lambda_y \left (y - \Delta y\right)\right) \cos\left(T_y y\right) \right]^2
 \end{align}
 que depenen dels paràmetres ajustats manualment sobre la corba empírca $h_x, h_y$, anomenats \emph{alçada}, $\Lambda_x, \Lambda_y$, als quals es farà referència per \emph{amplada} i $\Delta x, \Delta y$, que es designen per \emph{desplaçament}.

 Tenint en compte que, per al sensor CCD usat, $\SI{1}{\pixel} = \SI{6}{\micro\metre}$, per a l'obertura rectangular s'han obtingut els paràmetres que es detallen en la taula \ref{tab:Rect1-XY}.

\begin{figure}[H]
  \centering
  \subfloat[][]{
    \input{Figures/Rect1 - X}
  }
  \hspace{5mm}
  \subfloat[][]{
    \begin{tikzpicture}[]
\begin{axis}[
    width = 12cm,
    height = 7cm,
    enlarge x limits=0.10,
    enlarge y limits=0.10,
    %minor tick num = 3,
    xlabel={$y$ (\si{\milli\metre})},
    ylabel={$I$ (\si{\arbitrary})},
    xtick={-1.5,-1.0,...,1.5},
    %scaled x tick = {real:3.141516},
    ytick={0,50,...,300},
    %xticklabels={$-3$, $-2$, $-1$, $0$, $1$, $2$, $3$},
    minor xtick={-2,-1.9,...,2},
    minor ytick={-30,-20,...,330},
    xmin = -1.5, xmax = 1.5, extra x ticks ={},
    ymin = 0, ymax = 300,
    %extra y ticks ={4.18},
    %ymode = log,
    %log y ticks with fixed point,
    legend image post style={scale=0.5},
    legend pos=north west,
    /pgf/number format/.cd,
    1000 sep={},
    x tick label style={
    /pgf/number format/.cd,
    fixed,
    fixed zerofill,
    precision=1
  },
    y tick label style={
    /pgf/number format/.cd,
    fixed,
    fixed zerofill,
    precision=0
  },
]
\addplot[red, thick, scatter, scatter/classes={a={mark=diamond, red!80!, scale=0.5},b={mark=x,black,scale=1.2}}, scatter src=explicit symbolic
    ] plot []table[x = x, y = I, meta=class]{Dades/Rect1 - Col.csv};
\addplot[blue, thick, opacity=0.6, scatter, scatter/classes={a={mark=+, blue!80!, scale=0.5, opacity=0.6},b={mark=x,black,scale=1.2}}, scatter src=explicit symbolic
        ] plot []table[x = x, y = i, meta=class]{Dades/Rect1 - Col.csv};
\end{axis}
\end{tikzpicture}
\label{fig:Rect1-Y}

  }
\captionof{figure}{(\textsc{a}) Intensitats detectada i ajustada en funció de la posició sobre l'eix horitzontal central $X$ i (\textsc{b}) sobre l'eix vertical central $Y$ del sensor CCD, per a l'obertura rectangular.}
\label{fig:Rect1-XY}
\end{figure}



\appendix
\newpage
\section{Fórmules estadístiques}\label{sec:apendix}
\setcounter{equation}{0}
\renewcommand{\theequation}{\thesection.\arabic{equation}}
%\renewcommand{\thesubsection}{\thesection.\Roman{subsection}}
\subsection{Paràmetres centrals i de dispersió}
Donada una mostra de $N$ elements $\left\{x_1,\dots,x_N\right\}$, definim els següents paràmetres estadístics.
\begin{itemize}[label=--]
    \item Mitjana aritmètica \begin{equation}\label{est:mitjana}
            \langle x\rangle =\frac{1}{N} \sum_{i=1}^{N} x_i
        \end{equation}
    \item Desviació estàndard\footnote{S'ha fet servir la correcció de Bessel $\sqrt{N/(N-1)}$ de la desviació estàndard poblacional.} \begin{equation}\label{est:stdev}
            {\displaystyle \sigma_x ={\sqrt {{\frac {1}{N-1}}\sum _{i=1}^{N}(x_{i}-\langle x \rangle )^{2}}}}
        \end{equation}
    \item Error estàndard \begin{equation}\label{est:err}
   {\displaystyle \delta{x} ={\frac {\sigma_x}{\sqrt {N}}}}
\end{equation}

\end{itemize}
\subsection{Estimacions lineals}
Per al càlcul de les estimacions lineals s'usa la funció \texttt{ESTIMACION.LINEAL} del full de càlcul \texttt{Microsoft Excel}, que proporciona el pendent i l'ordenada a l'origen de la recta de regressió amb les seves corresponents incerteses, així com el coeficient de correlació $R^2$ i l'error estàndard de la regressió.
% El funcionament complet de la funció es pot consultar a \url{https://support.office.com/es-es/article/estimacion-lineal-funci\%C3\%B3n-estimacion-lineal-84d7d0d9-6e50-4101-977a-fa7abf772b6d}.

\subsection{Test \texorpdfstring{$\chi^2$}{x2}}
Donat un ajust lineal $y=A\,x+B$, amb incertesa en la variable dependent $\delta y$ i error estàndard de la regressió $\delta y_\textup{reg}$, es defineix el \emph{coeficient $\chi^2$} com
\begin{equation}
    \chi^2 = \nu\, \left(\frac{\delta y_\textup{reg}}{\delta y}\right)
\end{equation}
on $\nu$ es el nombre de graus de llibertat de l'ajust. Amb això, podem definir el \emph{coeficient reduït $\chi^2_\nu$} com
\begin{equation}
    \chi^2_\nu = \frac{\delta y_\textup{reg}}{\delta y}
\end{equation}
El valor d'aquest paràmetre ens indica la bondat de l'ajust realitzat. Tenim que
\begin{enumerate}[label=(\alph*)]
    \item si \underline{$\delta y_\textup{reg} \ll \delta y$} o \underline{$\chi^2_\nu \ll 1$}, l'ajust és acceptable i probablement s'hagi sobreestimat $\delta y$;
    \item si \underline{$\delta y_\textup{reg} \lesssim \delta y$} o \underline{$\chi^2_\nu \lesssim 1$}, l'ajust és acceptable; i
    \item si \underline{$\delta y_\textup{reg} \gg \delta y$} o \underline{$\chi^2_\nu \gg 1$} l'ajust no és acceptable.
\end{enumerate}

\end{document}
