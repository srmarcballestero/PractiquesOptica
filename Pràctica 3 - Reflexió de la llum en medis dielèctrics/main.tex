\documentclass[11pt,a4paper,catalan]{article}
\author{Marc Ballestero Ribó}

%___PAQUETS NECESSARIS___%
\usepackage[utf8]{inputenc}
\usepackage{amsthm,amssymb,amsmath,mathrsfs}
\usepackage[catalan]{babel}
\usepackage{wasysym}
\usepackage[margin=25mm]{geometry}
\usepackage[bottom]{footmisc}
\usepackage{enumerate}
\usepackage{array, boldline, makecell, booktabs, dcolumn}
\usepackage[svgnames, table]{xcolor}
\usepackage[all]{xy}
\usepackage{parskip}
\usepackage{float}
\usepackage{fancyhdr}
\usepackage{graphicx}
\usepackage{multirow,multicol}
\usepackage{pgfplots, tikz}
\usepackage{pgfplotstable}
\usepackage{graphics}
\usepackage{arydshln}
\usepackage{caption}
\usepackage{siunitx}
\usepackage{chngpage}
\usepackage{enumitem}
\usepackage{hyperref}
\hypersetup{
    colorlinks=true,
    linkcolor=black,
    filecolor=black,
    urlcolor=blue,
}

\sisetup{separate-uncertainty=true, exponent-product=\cdot}
\DeclareSIUnit\dioptre{D}
\DeclareSIUnit\arbitrary{u.a.}
\DeclareSIUnit\percent{\%}
\decimalpoint



%\sisetup{locale = FR}

%___NOUS TIPUS DE COLUMNES PER LES TAULES___%
\newcolumntype{L}[1]{>{\raggedright\let\newline\\\arraybackslash\hspace{0pt}}m{#1}}
\newcolumntype{C}[1]{>{\centering\let\newline\\\arraybackslash\hspace{0pt}}m{#1}}
\newcolumntype{R}[1]{>{\raggedleft\let\newline\\\arraybackslash\hspace{0pt}}m{#1}}
\newcolumntype{d}[1]{D{.}{.}{#1}}

%___COMANDAMENTS PERSONALITZATS___%
\providecommand{\keywords}[1]{\textsc{{Paraules clau---}} #1}
\newcommand{\diff}{\mathrm{d}}
\newcommand{\nind}{\noindent}
\newcommand{\res}[3]{$\left ( #1 \pm #2 \right ) \, \si{#3}$}
\newcommand{\figref}[1]{figura \ref{#1}}
\newcommand{\tabref}[1]{taula \ref{#1}}
\renewcommand\thesection{\arabic{section}}
\newcommand{\data}{\today}


%___CAPÇALERA I PEU DE PÀGINA___%
\pagestyle{fancy}
\headheight=16pt
\setlength{\headsep}{1cm}
\fancypagestyle{headings}{
\cfoot{\thepage}}
\lhead{\scshape Òptica \textup{2020-21}}
\rhead{\data \\ Marc Ballestero Ribó}

%___CONFIGURACIÓ DEL PGFPLOTS___%
\pgfplotsset{compat = 1.15}
\pgfplotsset{
    legend image with text/.style={
        legend image code/.code={%
            \node[anchor=center] at (0.3cm,0cm) {#1};
        }
    },
    log x ticks with fixed point/.style={
      xticklabel={
        \pgfkeys{/pgf/fpu=true}
        \pgfmathparse{exp(\tick)}%
        \pgfmathprintnumber[fixed relative, precision=0]{\pgfmathresult}
        \pgfkeys{/pgf/fpu=false}
      }
    },
    log y ticks with fixed point/.style={
      yticklabel={
        \pgfkeys{/pgf/fpu=true}
        \pgfmathparse{exp(\tick)}%
        \pgfmathprintnumber[fixed zerofill, precision=0]{\pgfmathresult}
        \pgfkeys{/pgf/fpu=false}
      }
  },
}
\usetikzlibrary{spy}

%___CONFIGURACIÓ DELS PEUS DE FIGURA___%
\captionsetup[figure]{justification = justified,labelfont={small,sc},textfont=small}
\captionsetup[table]{name = Taula, justification = justified,labelfont={small,sc},textfont=small}

%___DIRECTORI D'IMATGES___%
\graphicspath{ {./Imatges/} }




%___TÍTOL___%

\title{\vspace{-1.5cm} Òptica \\ \vspace{1mm} Pràctica 3. Reflexió de la llum en medis dielèctrics}
\author{Marc Ballestero Ribó - Grup D2}
\date{\data}


\begin{document}
\maketitle

\vspace{3mm}

\begin{multicols}{2}

 \section{Determinació visual de l'angle de Brewster} \label{sec:BrewsterVisual}

 Considerem un feix de llum natural monocromàtica que incideix sobre la superfície d'un medi dielèctric d'índex de refracció $n_\textup{t}$, provinent d'un medi d'índex $n_\textup{i}$, formant un angle $\theta_\textup{i}$ amb la direcció normal. La llei de la reflexió ens determina que l'angle que forma el raig reflectit amb la normal és $\theta_\textup{r} = \theta_\textup{i}$ i la llei d'Snell ens dóna el valor de l'angle de deflexió del raig transmès, segons $\sin{\theta_{r}} = \left({n_\textup{i}}/{n_\textup{t}}\right)\sin{\theta_\textup{i}}$.

 Segons les \emph{fórmules de Fresnel}, per a un material dielèctric, les amplituds incidents i reflectides del camp elèctric es relacionen segons
 \begin{equation} \label{eq:Fresnel}
  \frac{E_\perp^\textup{r}}{E_\perp} = -\frac{\sin\left(\theta_\textup{i} - \theta_\textup{t}\right)}{\sin\left(\theta_\textup{i} + \theta_\textup{t}\right)}, \hspace{1mm} \frac{E_\shortparallel^\textup{r}}{E_\shortparallel} = \frac{\tan\left(\theta_\textup{i} - \theta_\textup{t}\right)}{\tan\left(\theta_\textup{i} + \theta_\textup{t}\right)}
 \end{equation}
 on el subíndex $\perp$ fa referència a la component perpendicular al pla d'incidència i $\shortparallel$ a la paral·lela a l'esmentat pla.

 Quan l'angle d'incidència és tal que $\theta_\textup{i} + \theta_\textup{t} = \pi/2$, és senzill comprovar que s'anul·la la component $E_\shortparallel^\textup{r}$, és a dir, la llum reflectida queda polaritzada linealment en la direcció perpendicular al pla d'incidència. Aquest angle
 s'anomena \emph{angle de Brewster} i es denota $\theta_\textup{iB}$, i es pot comprovar que satisfà
 \begin{equation} \label{eq:tanBrewster}
  \tan{\theta_\textup{iB}} = \frac{n_\textup{t}}{n_\textup{i}}.
 \end{equation}

 L'objectiu d'aquesta secció és determinar l'angle de Brewster d'una làmina dielèctrica de vidre per tempteig visual. El sistema experimental consta d'una làmpada de sodi que dóna una llum que, en molt bona aproximació, es pot considerar monocromàtica de longitud $\SI{589}{\nano\metre}$, un col·limador que projecta rajos paral·lels sobre la mostra de material dielèctric, i un telescopi per a l'observació del feix reflectit, tot muntat sobre un goniòmetre. Tant sobre el col·limador com sobre el telescopi s'hi poden acoblar polaritzadors lineals. Anomenarem \emph{analitzador} al polarizador muntat sobre l'ocular del telescopi.

 Procedint per tempteig, trobem l'angle d'incidència pel qual el feix de llum reflectit s'anul·la totalment per a una determinada posició de l'analitzador sobre el telescopi. En aquesta situació, podem assegurar que l'angle d'incidència és l'angle de Brewster, i que l'eix de l'analitzador és paral·lel al pla d'incidència, la qual cosa ens permet determinar el zero d'aquest polaritzador.

 Podem estimar doncs, mitjançant el goniòmetre, que l'angle de Brewster de la mostra és $\theta_\textup{iB} = \ang{56;45;} \pm \ang{0;1;}$.

 Fent sevir \eqref{eq:tanBrewster}, i tenint en compte que l'índex de refracció de l'aire és $n_\textup{i} \approx 1.000$, podem calcular l'índex de refracció de la mostra segons
 \begin{equation*}
  n_\textup{t} = n_\textup{i} \, \tan{\theta_\textup{iB}}
 \end{equation*}
 d'on s'obté el valor $n_\textup{t} = \num{1.5253(9)}$, que pertany al rang d'índexs que presenta el vidre.

 \section{Mesura de l'angle de Brewster mitjançant la polarització} \label{sec:BrewsterPolaritzat}

 Havent determinat el zero de l'analitzador a la secció \ref{sec:BrewsterVisual}, podem obtenir un feix de llum monocromàtica polaritzada $\ang{45;;}$ respecte el pla d'incidència, acoblant el polaritzador a a la sortida del col·limador.

 Amb aquesta configuració, s'han realitzat mesures de l'angle $\alpha$ de l'eix de l'analitzador pel qual s'asoleix l'anul·lació de la llum, en funció de l'angle d'incidència del feix sobre la mostra. Amb això, s'han construït les dades de la taula \ref{tab:IncidenciaPolaritzador}.

 \begin{table}[H]
    \centering
    \footnotesize
    %\sisetup{separate-uncertainty=false}
    %\setlength{\tabcolsep}{6pt}
    \begin{tabular}{
            S[table-format=2.1]
            S[table-format=+2.1]
            S[table-format=1.3]
        }
    \toprule
        {$\theta_\textup{i} \pm 0.1$ (\si{\degree})} & {$\alpha \pm 0.1$ (\si{\degree})} & {$\left | \tan{\left(\alpha / \si{\degree}\right)}\right | \pm 0.002$} \\
    \midrule
    35.0	&	28.9	&	0.552	\\
    40.0	&	22.5	&	0.414	\\
    45.0	&	18.3	&	0.331	\\
    50.0	&	8.9	&	0.157	  \\
    52.0	&	6.1	&	0.107	  \\
    54.0	&	3.4	&	0.059	  \\
    56.0	&	-0.2	&	0.003	\\
    58.0	&	-0.6	&	0.010	\\
    60.0	&	-5.1	&	0.089	\\
    62.0	&	-7.0	&	0.123	\\
    65.0	&	-11.9	&	0.211	\\
    68.0	&	-17.5	&	0.315	\\
        \bottomrule
    \end{tabular}
    \captionof{table}{Dades experimentals i paràmetres calculats a partir d'aquestes.}
    \label{tab:IncidenciaPolaritzador}
    %\sisetup{separate-uncertainty=true}
\end{table}


 Així doncs, s'han representat els punts experimentals $\left | \tan{\left(\alpha / \si{\degree}\right)}\right |$ en funció de l'angle d'incidència en la figura \ref{fig:IncidenciaPolaritzador}. A més a més, s'ha realitzat l'ajust lineal\footnote{Per a l'ajust, s'ha obviat el valor absolut i s'ha tingut en compte el signe que prèn la tangent en funció de l'angle d'incidència.} $\tan{\left(\alpha/\si{\degree}\right)}\left(\theta_\textup{i}\right)$, els paràmetres del qual es detallen en la taula \ref{fig:AjustPolaritzador}.

 \begin{figure}[H]
\centering
\begin{tikzpicture}[scale=0.85]
    \begin{axis}[
        enlarge x limits=0.10,
        enlarge y limits=0.10,
        %minor tick num = 3,
        xlabel={$\theta_\textup{i}$ (\si{\degree})},
        ylabel={$\tan{\left(\alpha/\si{\degree}\right)}$},
        xtick = {30, 40,...,70},
        %scaled x tick = {real:3.141516},
        ytick={-0.4,-0.2,...,0.6},
        %xticklabels={$0$, $\frac{1}{4}\pi$, $\frac{1}{2}\pi$, $\frac{3}{4}\pi$, $\pi$, $\frac{5}{4}\pi$, $\frac{3}{2}\pi$, $\frac{7}{4}\pi$, $2\pi$},
        minor xtick={20,22.5,...,80},
        minor ytick={-0.5,-0.45,...,0.7},
        xmin = 30, xmax = 70, extra x ticks ={},
        ymin = -0.4, ymax = 0.6,
        %extra y ticks ={4.18},
        %ymode = log,
        %log y ticks with fixed point,
        legend image post style={scale=1},
        legend pos=south west,
        /pgf/number format/.cd,
        1000 sep={},
        %xticklabel style={xshift=-2pt},
        x tick label style={
        /pgf/number format/.cd,
        fixed,
        fixed zerofill,
        precision=0,
      },
        y tick label style={
        /pgf/number format/.cd,
        fixed,
        fixed zerofill,
        precision=1
      },
    ]

    %\addplot[samples=200,domain=0:8000, color=orange] {0.00606404*(x)+22.434048};
    %\addlegendentry{\footnotesize $I = I_0 \,\cos^2{\left(\beta - \alpha_0\right)}$}
    %\addlegendentry{\tiny $T = 0.006064 \, Q + 22.43$, $R^2 = 0.99990$}
    %\addlegendimage{legend image with text=}
    %\addlegendentry{\footnotesize $R^2=0.9993$}
    \addplot[scatter, only marks, scatter/classes={a={mark=*, black, scale=1}, b={mark=o, blue, scale=1}}, scatter src=explicit symbolic
        ] plot []table[x = t, y = tana, x error = dt, y error = dtana, meta=class]{Dades/Alpha_Theta.csv};
    \addlegendentry{\footnotesize $\left | \tan{\left(\alpha/\si{\degree}\right)} \right |$}
    \addlegendentry{\footnotesize $\tan{\left(\alpha/\si{\degree}\right)}$}
    \end{axis}
\end{tikzpicture}
\captionof{figure}{Intensitat detectada pel díode en funció de l'angle de l'escala de l'analitzador, amb la representació de les mitjanes i desviacions estàndards corresponents.}
\label{fig:IncidenciaPolaritzador}
\end{figure}


 
\end{multicols}

\appendix
\newpage
\section{Fórmules estadístiques}\label{sec:apendix}
\setcounter{equation}{0}
\renewcommand{\theequation}{\thesection.\arabic{equation}}
%\renewcommand{\thesubsection}{\thesection.\Roman{subsection}}
\subsection{Paràmetres centrals i de dispersió}
Donada una mostra de $N$ elements $\left\{x_1,\dots,x_N\right\}$, definim els següents paràmetres estadístics.
\begin{itemize}[label=--]
    \item Mitjana aritmètica \begin{equation}\label{est:mitjana}
            \langle x\rangle =\frac{1}{N} \sum_{i=1}^{N} x_i
        \end{equation}
    \item Desviació estàndard\footnote{S'ha fet servir la correcció de Bessel $\sqrt{N/(N-1)}$ de la desviació estàndard poblacional.} \begin{equation}\label{est:stdev}
            {\displaystyle \sigma_x ={\sqrt {{\frac {1}{N-1}}\sum _{i=1}^{N}(x_{i}-\langle x \rangle )^{2}}}}
        \end{equation}
    \item Error estàndard \begin{equation}\label{est:err}
   {\displaystyle \delta{x} ={\frac {\sigma_x}{\sqrt {N}}}}
\end{equation}

\end{itemize}
\subsection{Estimacions lineals}
Per al càlcul de les estimacions lineals s'usa la funció \texttt{ESTIMACION.LINEAL} del full de càlcul \texttt{Microsoft Excel}, que proporciona el pendent i l'ordenada a l'origen de la recta de regressió amb les seves corresponents incerteses, així com el coeficient de correlació $R^2$ i l'error estàndard de la regressió.
% El funcionament complet de la funció es pot consultar a \url{https://support.office.com/es-es/article/estimacion-lineal-funci\%C3\%B3n-estimacion-lineal-84d7d0d9-6e50-4101-977a-fa7abf772b6d}.

\subsection{Test \texorpdfstring{$\chi^2$}{x2}}
Donat un ajust lineal $y=A\,x+B$, amb incertesa en la variable dependent $\delta y$ i error estàndard de la regressió $\delta y_\textup{reg}$, es defineix el \emph{coeficient $\chi^2$} com
\begin{equation}
    \chi^2 = \nu\, \left(\frac{\delta y_\textup{reg}}{\delta y}\right)
\end{equation}
on $\nu$ es el nombre de graus de llibertat de l'ajust. Amb això, podem definir el \emph{coeficient reduït $\chi^2_\nu$} com
\begin{equation}
    \chi^2_\nu = \frac{\delta y_\textup{reg}}{\delta y}
\end{equation}
El valor d'aquest paràmetre ens indica la bondat de l'ajust realitzat. Tenim que
\begin{enumerate}[label=(\alph*)]
    \item si \underline{$\delta y_\textup{reg} \ll \delta y$} o \underline{$\chi^2_\nu \ll 1$}, l'ajust és acceptable i probablement s'hagi sobreestimat $\delta y$;
    \item si \underline{$\delta y_\textup{reg} \lesssim \delta y$} o \underline{$\chi^2_\nu \lesssim 1$}, l'ajust és acceptable; i
    \item si \underline{$\delta y_\textup{reg} \gg \delta y$} o \underline{$\chi^2_\nu \gg 1$} l'ajust no és acceptable.
\end{enumerate}

\end{document}
