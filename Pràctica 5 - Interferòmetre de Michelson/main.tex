\documentclass[11pt,a4paper,catalan]{article}
\author{Marc Ballestero Ribó}

%___PAQUETS NECESSARIS___%
\usepackage[utf8]{inputenc}
\usepackage{amsthm,amssymb,amsmath,mathrsfs}
\usepackage[catalan]{babel}
\usepackage{wasysym}
\usepackage[margin=25mm]{geometry}
\usepackage[bottom]{footmisc}
\usepackage{enumerate}
\usepackage{array, boldline, makecell, booktabs, dcolumn}
\usepackage[svgnames, table]{xcolor}
\usepackage[all]{xy}
\usepackage{parskip}
\usepackage{float}
\usepackage{fancyhdr}
\usepackage{graphicx}
\usepackage{multirow,multicol}
\usepackage{pgfplots, tikz}
\usepackage{pgfplotstable}
\usepackage{graphics}
\usepackage{arydshln}
\usepackage{caption}
\usepackage{siunitx}
\usepackage{chngpage}
\usepackage{enumitem}
\usepackage{hyperref}
\hypersetup{
    colorlinks=true,
    linkcolor=black,
    filecolor=black,
    urlcolor=blue,
}

\sisetup{separate-uncertainty=true, exponent-product=\cdot}
\DeclareSIUnit\dioptre{D}
\DeclareSIUnit\arbitrary{u.a.}
\DeclareSIUnit\percent{\%}
\decimalpoint



%\sisetup{locale = FR}

%___NOUS TIPUS DE COLUMNES PER LES TAULES___%
\newcolumntype{L}[1]{>{\raggedright\let\newline\\\arraybackslash\hspace{0pt}}m{#1}}
\newcolumntype{C}[1]{>{\centering\let\newline\\\arraybackslash\hspace{0pt}}m{#1}}
\newcolumntype{R}[1]{>{\raggedleft\let\newline\\\arraybackslash\hspace{0pt}}m{#1}}
\newcolumntype{d}[1]{D{.}{.}{#1}}

%___COMANDAMENTS PERSONALITZATS___%
\providecommand{\keywords}[1]{\textsc{{Paraules clau---}} #1}
\newcommand{\diff}{\mathrm{d}}
\newcommand{\nind}{\noindent}
\newcommand{\res}[3]{$\left ( #1 \pm #2 \right ) \, \si{#3}$}
\newcommand{\figref}[1]{figura \ref{#1}}
\newcommand{\tabref}[1]{taula \ref{#1}}
\renewcommand\thesection{\arabic{section}}
\newcommand{\data}{\today}


%___CAPÇALERA I PEU DE PÀGINA___%
\pagestyle{fancy}
\headheight=16pt
\setlength{\headsep}{1cm}
\fancypagestyle{headings}{
\cfoot{\thepage}}
\lhead{\scshape Òptica \textup{2020-21}}
\rhead{\data \\ Marc Ballestero Ribó}

%___CONFIGURACIÓ DEL PGFPLOTS___%
\pgfplotsset{compat = 1.15}
\pgfplotsset{
    legend image with text/.style={
        legend image code/.code={%
            \node[anchor=center] at (0.3cm,0cm) {#1};
        }
    },
    log x ticks with fixed point/.style={
      xticklabel={
        \pgfkeys{/pgf/fpu=true}
        \pgfmathparse{exp(\tick)}%
        \pgfmathprintnumber[fixed relative, precision=0]{\pgfmathresult}
        \pgfkeys{/pgf/fpu=false}
      }
    },
    log y ticks with fixed point/.style={
      yticklabel={
        \pgfkeys{/pgf/fpu=true}
        \pgfmathparse{exp(\tick)}%
        \pgfmathprintnumber[fixed zerofill, precision=0]{\pgfmathresult}
        \pgfkeys{/pgf/fpu=false}
      }
  },
}
\usetikzlibrary{spy}

%___CONFIGURACIÓ DELS PEUS DE FIGURA___%
\captionsetup[figure]{justification = justified,labelfont={small,sc},textfont=small}
\captionsetup[table]{name = Taula, justification = justified,labelfont={small,sc},textfont=small}

%___DIRECTORI D'IMATGES___%
\graphicspath{ {./Imatges/} }




%___TÍTOL___%

\title{\vspace{-1.5cm} Òptica \\ \vspace{1mm} Pràctica 5. Interferòmetre de Michelson}
\author{Marc Ballestero Ribó - Grup D2}
\date{\data}


\begin{document}
\maketitle

\begin{abstract}
 \nind
\end{abstract}

\section{Punt d'equidistància} \label{sec:puntEqui}
El primer pas que s'ha realitzat és la calibració i posada a punt de l'interferòmetre, tot regulant finament l'orientació del mirall $\textup{E}_2$ per tal d'aconseguir-ne la perpendicularitat perfecta amb $\textup{E}_1$. En aquestes condicions, amb el sistema òptic centrat, es poden observar anells d'interferència centrats.

Ara, es proposa estudiar qualitativament el comportament interferomètric del sistema quan es modifica la longitud d'un dels braços, mitjançant la regulació micromètrica de la posició del mirall $\textup{E}_1$. S'observa que, en un desplaçament continu, apareixen o desapareixen anells d'interferència en funció de com es varia la longitud del braç.

Sigui $d$ la diferència de longitud dels dos braços, que s'expressa
\begin{equation}
 d = \textup{LE}_2 - \textup{LE}_1
\end{equation}
on $\textup{LE}_i$ és la distància entre la làmina separadora $\textup{L}$ i el mirall $\textup{E}_i$. Siguin $\textup{S}'$ i $\textup{S}''$ les imatges de la font en els miralls $\textup{E}_1$ i $\textup{E}_2$, respectivament. La diferència de camí òptic, observada en $\textup{O}$, entre dos rajos provinents de $\textup{S}'$ i $\textup{S}''$ amb un angle $\theta$, ve donada per
\begin{equation}
 \Delta = 2 d\cos{\theta}
\end{equation}
d'on es dedueix que
\begin{itemize}[label = --]
 \item s'observaran màxims quan $2 d\cos{\theta} = k \lambda$, i
 \item s'observaran mínims quan $2 d\cos{\theta} = \left(2k + 1 \right) \frac{\lambda}{2}$,
\end{itemize}
on $k\in\mathbb{N}$ i $\lambda$ és la longitud d'ona dels rajos.

Per un anell\footnote{D'ara en endavant, el terme \emph{anell} fa referència a una zona d'interferència constructiva.} fixat, el producte $k\lambda$ és constant, d'on, tenint en compte la relació entre el camí òptic i la longitud d'ona en un màxim, si la distància $d$ augmenta, el terme $\cos{\theta}$ ha de disminuir. Tenint en compte que per $\theta \in \left [0, \pi/2 \right ]$, la funció $\cos{\theta}$ és decreixent, es dedueix que en aquestes condicions $\theta$ augmenta. És per això que, quan s'allunya el mirall $\textup{E}_1$, s'observa l'aparició d'anells d'interferència pel centre de la imatge.

De manera simètrica, es pot explicar la desaparició d'anells pel centre quan es disminueix la distància $d$.


\section{Longituds d'ona} \label{sec:lambdes}
En aquest apartat es calculen les longituds d'ona de les diferents fonts de l'interferòmetre a partir de les mesures realitzades en el patró interferomètric i en el propi sistema òptic.

Considerant el punt d'incidència normal, es té $\theta = \SI{0}{\degree} \Rightarrow \cos{\theta} = \num{1}$, i per diferències $d_1$ i $d_2$ de longitud dels braços, es té
\begin{equation} \label{eq:lambda1}
 \left.
 \begin{matrix}
  2d_1 = k_1 \lambda \\
  2d_2 = k_2 \lambda
 \end{matrix}
 \right \}
 \Longrightarrow 2 \left (d_1 - d_2 \right ) = \left (k_1 - k_2 \right) \lambda = n \lambda
\end{equation}
on $n \in \mathbb{Z}$ és el nombre de màxims que apareixen o desapareixen pel centre de la imatge, essent $n>0$ si n'apareixen i $n<0$ si en desapareixen.

Vist que el cargol micromètric que regula el desplaçament del mirall $\textup{E}_1$ té una reducció mecànica $1:5$, es té que $d_1 - d_2 = \left (D_1 - D_2 \right) / 5$, on $D_i$ són els valors mesurats en l'escala del micròmetre. Finalment, de \eqref{eq:lambda1} es desprèn que
\begin{equation} \label{eq:lambda}
 \lambda = \frac{2}{5n}\left(D_1 - D_2 \right)
\end{equation}

\subsection{Làser díode} \label{sec:laser}
Incorporant com a font del sistema òptic un làser díode expandit amb una lent, s'ha obtingut que, entre $D_1 = \SI{4.320(5)}{\milli\metre}$ i $D_2 = \SI{4.230(5)}{\milli\metre}$ apareixen $n = \num{55(2)}$ anells d'interferència\footnote{En la secció \ref{sec:errors} es discuteix tot allò relatiu a la incertesa que presenten les mesures experimentals.}. L'error en les distàncies s'ha près com la meitat de la resolució de l'escala del micròmetre.

Aplicant \eqref{eq:lambda} i propagant les incerteses corresponents, s'obté que $\lambda_\textup{laser} = \SI{650 \pm 60}{\nano\metre}$, valor que és compatible amb el d'un làser díode de color vermell.

\subsection{Làmpada de sodi i mesura del doblet del sodi} \label{sec:Na}
En aquest cas s'ha il·luminat el sistema mitjançant una làmpada de sodi, que emet en dues longituds d'ona molt properes $\lambda_\textup{Na}$ i $\lambda_\textup{Na}'$, anomenades \emph{doblet del sodi}.

Per a mesurar la longitud d'ona de la font a partir del patró interferomètric, s'ha considerat que $\lambda_\textup{Na} \approx \lambda_\textup{Na}'$ i s'ha procedit de manera anàloga a la secció anterior, realitzant dues mesures en regions de separació del mirall diferents per a més precisió.

\begin{itemize} [label = --]
 \item \textbf{Mesura 1}. S'ha obtingut que entre $D_1 = \SI{6.720(5)}{\milli\metre}$ i $D_2 = \SI{6.800(5)}{\milli\metre}$ desapereixien $n = \num{-54(2)}$ anells, per tant, aplicant \eqref{eq:lambda} s'obté que $\lambda_{\textup{Na},1} = \SI{590 \pm 60}{\nano\metre}$.
 \item \textbf{Mesura 2}. S'ha obtingut que entre $D_1 = \SI{8.020(5)}{\milli\metre}$ i $D_2 = \SI{7.940(5)}{\milli\metre}$ apareixien $n = \num{58(2)}$ anells, per tant, aplicant \eqref{eq:lambda} s'obté que $\lambda_{\textup{Na},2} = \SI{550 \pm 50}{\nano\metre}$.
\end{itemize}

Així doncs, la longitud d'ona de la font de sodi s'ha estimat com la mitjana de les dues mesures realitzades, calculada segons \eqref{est:mitjana}, obtenint que $\lambda_\textup{Na,av} = \SI{570 \pm 40}{\nano\metre}$, valor que es compatible amb les longituds d'ona del doblet de sodi\footnote{\textsc{Font:} NIST. \textit{Persistent Lines of Neutral Sodium (Na I)}. National Institute of Standards and Technology - NIST. \url{https://physics.nist.gov/PhysRefData/Handbook/Tables/sodiumtable3.htm}.} $\lambda_\textup{Na} = \SI{589.0}{\nano\metre}$ i $\lambda_\textup{Na}' = \SI{589.6}{\nano\metre}$ dins del marge d'incertesa corresponent.

Mitjançant l'interferòmetre, també es pot mesurar la separació en $\lambda$ del doblet del sodi, aprofitant el fet que les dues longituds d'ona dominants del sodi són molt properes i donen lloc a pulsacions en el contrast de la imatge. Considerant $\lambda_\textup{Na} < \lambda_\textup{Na}'$, es té que per a dues posicions de mínim de contrast consecutives, $d_1$ i $d_2$
\begin{equation} \label{eq:Dlambda}
 \left.
 \begin{matrix}
  2d_1 = k_1 \lambda_\textup{Na} = k_1' \lambda_\textup{Na}' \\
  2d_2 = k_2 \lambda_\textup{Na} = k_2' \lambda_\textup{Na}'
 \end{matrix}
 \right \}
 \Longrightarrow \Delta \lambda_\textup{Na} := \lambda_\textup{Na} - \lambda_\textup{Na}' = \frac{\lambda_\textup{Na} \lambda_\textup{Na}'}{2 \left(d_1 - d_2\right)} \approx \frac{5 \lambda_\textup{Na,av} ^2}{2\left(D_1 - D_2\right)}
\end{equation}
on s'ha aproximat el producte de les longituds d'ona pel quadrat de la longitud d'ona mitjana obtinguda anteriorment.

S'han realitzat dues mesures per a major precisió.
\begin{itemize} [label = --]
 \item \textbf{Mesura 1}. S'ha obtingut que en $D_1 = \SI{7.895(5)}{\milli\metre}$ i $D_2 = \SI{6.025(5)}{\milli\metre}$ hi ha mínims de contrast consecutius, d'on, aplicant \eqref{eq:Dlambda} es té $\Delta \lambda_{\textup{Na},1} = \SI{0.43(6)}{\nano\metre}$.
 \item \textbf{Mesura 2}. S'ha obtingut que en $D_2 = \SI{6.025(5)}{\milli\metre}$ i $D_3 = \SI{4.630(5)}{\milli\metre}$ hi ha mínims de contrast consecutius, d'on, aplicant \eqref{eq:Dlambda} es té $\Delta \lambda_{\textup{Na},2} = \SI{0.58(8)}{\nano\metre}$.
\end{itemize}

Finalment, la separació del doblet del sodi s'ha estimat com la mitjana entre els dos valors obtinguts, calculada segons \eqref{est:mitjana}, obtenint que $\Delta \lambda_\textup{Na,av} = \SI{0.51(5)}{\nano\metre}$, que és compatible amb l'obtinguda amb els valors de la bibliografia $\Delta \lambda_\textup{Na} = \SI{0.6}{\nano\metre}$, ja que
\begin{equation*}
 \SI{0.09}{\nano\metre} = \left | \Delta\lambda_\textup{Na,av} - \Delta\lambda_\textup{Na} \right | \leq 2\,\delta\left(\Delta\lambda_\textup{Na,av}\right) = \SI{0.10}{\nano\metre}
\end{equation*}

\section{Anàlisi de les fonts d'error experimental} \label{sec:errors}
En aquesta secció es discuteixen els aspectes referents a les diverses fonts d'error presents en l'experiment realitzat.

Primer de tot, cal considerar l'error intrínsec a la calibració i centrat del sistema òptic; es tracta d'una font d'error sistemàtic present durant tot el procediment experimental, que si més no, no es pot quantificar, però si que cal tenir en compte a l'hora d'analitzar els resultats obtingts.

Més enllà d'això, es té que les mesures de distància $D_i$ en l'escala del micròmetre presenten un error sistemàtic associat a la resolució de l'instrument, que s'ha estimat com la meitat de la distància entre les marques més petites de l'escala graduada de l'instrument, vist que en molts casos els punts experimentals cauen entre dues marques consecutives. La incertesa és, doncs, $\delta D_i = \SI{0.005}{\milli\metre} = \SI{5e3}{\nano\metre}$.

Finalment, el nombre d'anells d'interferència que apareixen o desapareixen pel centre de la pantalla d'observació quan es modifica la distància relativa dels braços de l'interferòmetre també presenta una incertesa experimental associada al fet que, una variació molt petita de $d$ sovint provoca un canvi sobtat en $n$, la qual cosa és una font d'errors de comptatge. Així doncs, s'ha estimat que $\delta n = \num{2}$ anells.

Si s'aplica l'algorisme de propagació d'incerteses no correlacionades a l'equació \eqref{eq:lambda}, s'obté que
\begin{equation} \label{eq:dlambda}
 \delta \lambda = \frac{2}{5\left |n\right |} \sqrt{\left [\left (D_1 - D_2 \right) \frac{\delta n}{n}\right] ^2 + \left (\delta D_1\right)^2 + \left(\delta D_2\right)^2}
\end{equation}
.

Fixat $\delta D_i = \SI{5e3}{\nano\metre}$ es té que
\begin{equation*}
 \delta \lambda = \frac{2}{5\left |n\right |} \sqrt{\left [\left (D_1 - D_2 \right) \frac{\delta n}{n}\right] ^2 + \SI{5e7}{\nano\metre\squared}}, \hspace{3mm} \left[\delta \lambda\right] = \si{\nano\metre}
\end{equation*}
d'on, vist que $\left [\left (D_1 - D_2 \right) \frac{\delta n}{n}\right]^2 \ll \SI{5e7}{\nano\metre\squared}$, l'error introduït si hi ha un error de comptatge en els anells d'interferència és poc significatiu si es té en compte l'error en la mesura del micròmetre.

Els valors numèrics dels errors en la longitud d'ona i els aspectes relatius a la compatibilitat dels resultats experimentals amb els valors experimentals ja s'han detallat a la secció \ref{sec:lambdes}.

D'altra banda, propagant incerteses en l'equació \eqref{eq:Dlambda}, s'obté que l'error en la separació del doblet del sodi ve donat per
\begin{equation} \label{eq:dDlambda}
  \delta\left(\Delta\lambda_\textup{Na,av}\right) = \frac{5\lambda_\textup{Na,av}}{D_1 - D_2} \sqrt{\left (\delta \lambda_\textup{Na,av}\right)^2 + \left[\frac{\lambda_\textup{Na,av}^2}{2\left(D_1 - D_2 \right)}\, \delta \left(D_1 - D_2 \right)\right]^2}
\end{equation}
amb la qual cosa, atès que l'error en el comptatge d'anells no afecta significativament a l'error en la longitud d'ona, $\delta\left(\Delta\lambda_\textup{Na,av}\right)$ tampoc es veu substancialment afectat.

Igualment, el valor numèric del càlcul corresponent ja s'ha exposat a la secció \ref{sec:lambdes}.




\appendix
\newpage
\section{Fórmules estadístiques}\label{sec:apendix}
\setcounter{equation}{0}
\renewcommand{\theequation}{\thesection.\arabic{equation}}
%\renewcommand{\thesubsection}{\thesection.\Roman{subsection}}
\subsection{Paràmetres centrals i de dispersió}
Donada una mostra de $N$ elements $\left\{x_1,\dots,x_N\right\}$, definim els següents paràmetres estadístics.
\begin{itemize}[label=--]
    \item Mitjana aritmètica \begin{equation}\label{est:mitjana}
            \langle x\rangle =\frac{1}{N} \sum_{i=1}^{N} x_i
        \end{equation}
    \item Desviació estàndard\footnote{S'ha fet servir la correcció de Bessel $\sqrt{N/(N-1)}$ de la desviació estàndard poblacional.} \begin{equation}\label{est:stdev}
            {\displaystyle \sigma_x ={\sqrt {{\frac {1}{N-1}}\sum _{i=1}^{N}(x_{i}-\langle x \rangle )^{2}}}}
        \end{equation}
    \item Error estàndard \begin{equation}\label{est:err}
   {\displaystyle \delta{x} ={\frac {\sigma_x}{\sqrt {N}}}}
\end{equation}

\end{itemize}
\subsection{Estimacions lineals}
Per al càlcul de les estimacions lineals s'usa la funció \texttt{ESTIMACION.LINEAL} del full de càlcul \texttt{Microsoft Excel}, que proporciona el pendent i l'ordenada a l'origen de la recta de regressió amb les seves corresponents incerteses, així com el coeficient de correlació $R^2$ i l'error estàndard de la regressió.
% El funcionament complet de la funció es pot consultar a \url{https://support.office.com/es-es/article/estimacion-lineal-funci\%C3\%B3n-estimacion-lineal-84d7d0d9-6e50-4101-977a-fa7abf772b6d}.

\subsection{Test \texorpdfstring{$\chi^2$}{x2}}
Donat un ajust lineal $y=A\,x+B$, amb incertesa en la variable dependent $\delta y$ i error estàndard de la regressió $\delta y_\textup{reg}$, es defineix el \emph{coeficient $\chi^2$} com
\begin{equation}
    \chi^2 = \nu\, \left(\frac{\delta y_\textup{reg}}{\delta y}\right)
\end{equation}
on $\nu$ es el nombre de graus de llibertat de l'ajust. Amb això, podem definir el \emph{coeficient reduït $\chi^2_\nu$} com
\begin{equation}
    \chi^2_\nu = \frac{\delta y_\textup{reg}}{\delta y}
\end{equation}
El valor d'aquest paràmetre ens indica la bondat de l'ajust realitzat. Tenim que
\begin{enumerate}[label=(\alph*)]
    \item si \underline{$\delta y_\textup{reg} \ll \delta y$} o \underline{$\chi^2_\nu \ll 1$}, l'ajust és acceptable i probablement s'hagi sobreestimat $\delta y$;
    \item si \underline{$\delta y_\textup{reg} \lesssim \delta y$} o \underline{$\chi^2_\nu \lesssim 1$}, l'ajust és acceptable; i
    \item si \underline{$\delta y_\textup{reg} \gg \delta y$} o \underline{$\chi^2_\nu \gg 1$} l'ajust no és acceptable.
\end{enumerate}

\end{document}
